\chapter{Supergravity and Higher Dimensions}
\label{ch:supergravity}

In this thesis, our main focus is to understand solutions of supergravity theories containing planar symmetric Killing horizons and their corresponding thermodynamics. In the previous chapters, we have discussed how starting with a Lagrangian containing some matter content and an ansatz imposing a set of symmetries on the manifold, allows us to derive a metric describing the spacetime of the solution. From this perspective, the role of supergravity is to suggest a Lagrangian as a starting point from which we build our solutions in Chapter \ref{ch:planarstu}.

This chapter aims to introduce supergravity from the perspective of extending the possible physical models we consider as a general relativist, while also giving a backbone to some of the various comments and calculations that relate our four-dimensional solutions to contemporary research in string theory and higher-dimensional supergravity. 

We begin with Section \ref{sec:supergravity}, introducing supergravity from the perspective of the supersymmetry algebra and use this to define the matter content we would expect in a physical theory, focusing on $\N = 2$ supergravity in four dimensions. In Section \ref{sec:electromagneticduality}, we take a brief detour and discuss the electromagnetic duality and its generalisation, which appears in $\N = 2$ vector multiplets. This will be particularly important when we consider the thermodynamics of our charged solutions derived in Chapter \ref{ch:triplewick}. In Section \ref{sec:dimreduction}, we introduce Kaluza-Klein reduction, which is an incredibly useful tool for representing higher-dimensional supergravity in lower dimensions. From a phenomenological perspective, the process of dimensional reduction allows us to understand higher-dimensional theories such as string theory and M-theory in a way that matches our experience of a universe in four dimensions. In Section \ref{sec:cmap}, the c-map is introduced, which is a crucial result in finding non-extremal solutions of supergravity in four dimensions. The chapter concludes with Section \ref{sec:pbranes}, which gives an overview of higher-dimensional supergravity, aiming to introduce the reader to $p$-branes and their relationship with black hole solutions.

\section{\titleN$ = 2$ supergravity}
\label{sec:supergravity}

In this section, we provide an introduction to supergravity sufficient to motivate the supergravity actions which serve as a starting point from which we derive planar symmetric solutions. We begin in Section \ref{sec:supersymmetry} by motivating supersymmetry as the extension of the Poincar\'e algebra and study the massive and massless representations following \cite{Wess:1992cp}. In Section \ref{sec:supergravitylag}, we then specify to $\N = 2$ supergravity in four dimensions and present the Lagrangian of the bosonic content of $\N = 2$, two-derivative supergravity coupled to $n_V$ vector multiplets. 

\subsection{Supersymmetry algebra}
\label{sec:supersymmetry}

The laws of physics are understood to be invariant under translations (generated by the momentum operator $P_\mu$) and the Lorentz transformations (boosts and rotations, generated by $M_{\mu \nu}$). From these generators, one can build the Lie algebra of the Poincar\'e group
\begin{equation}
\label{eq:poincaregroup}
\begin{aligned}
	&[P_\mu, P_\nu] = 0, \\
	&[M_{\mu \nu}, P_\rho] = i\left(\eta_{\mu \rho} P_\nu - \eta_{\nu \rho} P_\mu \right), \\
	&[M_{\mu \nu}, M_{\rho \sigma}] = i\left(\eta_{\mu \rho} M_{\nu \sigma} - \eta_{\mu \sigma} M_{\nu \rho} - \eta_{\nu \rho} M_{\mu \sigma} + \eta_{\nu \sigma} M_{\mu \rho} \right).
\end{aligned}
\end{equation}
Supersymmetry comes naturally from the question of whether we can include additional operators which extend the Lie algebra. In 1967, Coleman and Mandula \cite{Coleman:1967ad} proved their no-go theorem that the most general \emph{bosonic} symmetries of the S-matrix must commute with the Poincar\'e algebra if we wish to maintain non-zero scattering amplitudes, \ie these additional generators transform as scalars. Of course, a no-go theorem is only as strong as its assumptions, and in 1971, Golfand and Likhtman were able to extended the algebra by including anti-commuting, \emph{fermionic} generators \cite{Golfand:1971iw}. The inclusion of fermionic generators generalises the Lie algebra to a graded Lie algebra, which is defined by
\begin{equation*}
	\Op_a \Op_b - (-1)^{\chi_a \chi_b} \Op_b \Op_a = i C^{c}_{ab} \Op_e,
\end{equation*}
for operators $\Op_a$, where $\chi_a = 0$ for \emph{bosonic} generators and $\chi_a = 1$ for \emph{fermionic} generators, and $C^{c}_{ab}$ are the structure constants. In 1975, Haag, Lopuszanski and Sohnius generalised Coleman and Mandula's no-go theorem stating that non-trivial quantum field theories have the super Poincar\'e alebgra as the most general algebra, and any additional symmetries will commute as internal scalars \cite{Haag:1974qh}.

The super Poincar\'e algebra is the extension of \eq{poincaregroup} which includes the additional fermionic generators, also known as \emph{supercharges} which we take to be Weyl spinors, transforming in the fundamental representation of SL$(2,\mathbb{C})$ and its complex conjugate. They are denoted by:
\begin{equation*}
	Q^A_{\alpha}, \; \bar{Q}^A_{\dot{\alpha}}, \quad \alpha, \dot{\alpha} \in \{1,2\}, \quad A \in \{1, \ldots, \N\},
\end{equation*}
where the indices $\alpha, \dot{\alpha}$ are spinorial indices, raised and lowered with $\delta_{\alpha \dot{\beta}}$, and $\N$ counts `the amount of supersymmetry'; more precisely, we say that a theory has $4\N$ real supercharges. The generators are related by $\bar{Q}^A_{\dot{\alpha}} = \tensor{\epsilon}{_{\dot{\alpha}}^\beta} (Q^A_\beta)^\star$.\footnote{If we consider the Weyl spinors as independent, we have four complex degrees of freedom, and $\N = 2$ supersymmetry realised in our algebra. By asserting the Weyl spinors are each other's complex conjugates, we reduce to two complex, or four real degrees of freedom which can be packaged together into a single Majorana spinor or the two related Weyl spinors introduced in the main text.} Minimal supersymmetry is for $\N = 1$, and maintaining that the highest spin states have spin-2 requires $\N \leq 8$ \cite{Nahm:1977tg}. 

Qualitatively, we can think of these fermionic operators generating a fermion from a boson and vice versa:
\begin{equation*}
	Q \ket{\text{fermion}} = \ket{\text{boson}}, \qquad Q \ket{\text{boson}} = \ket{\text{fermion}}.
\end{equation*}
These generators extend the Poincar\'e algebra with the following (anti-)commutation relations
\begin{equation}
\label{eq:superpoincaregroup}
\begin{aligned}
	&[P_\mu, Q_\alpha^A] = 0, \\
	&[M_{\mu \nu}, Q_\alpha^A] = - \frac{i}{2} \tensor{(\sigma_{\mu \nu})}{_\alpha^\beta} \tensor{Q}{^A_\beta}, \\
	&[M_{\mu \nu}, \bar{Q}_{\dot{\alpha}}^A] = - \frac{i}{2} \tensor{(\bar{\sigma}_{\mu \nu})}{_{\dot{\alpha}}^{\dot{\beta}}} \tensor{\bar{Q}}{^A_{\dot{\beta}}}, \\
	&\{ Q_\alpha^A,  \bar{Q}_{\dot{\beta}}^B \} = 2 \delta^{AB} (\sigma_\mu)_{\alpha \dot{\beta}} P^\mu ,\\
	&\{ Q_\alpha^A,Q_\beta^B \} = \epsilon_{\alpha \beta} Z^{AB} .
\end{aligned}
\end{equation}
The explicit form of the Pauli matrices is
\begin{equation*}
	\sigma_0 = \pmat{1}{0}{0}{1}, \quad 
	\sigma_1 = \pmat{0}{1}{1}{0}, \quad
	\sigma_2 = \pmat{0}{-i}{i}{0}, \quad
	\sigma_3 = \pmat{1}{0}{0}{-1},
\end{equation*}
and $\sigma_{\mu \nu} = \half \sigma_{[\mu} \sigma_{\nu]}$. The relations \eq{superpoincaregroup} together with \eq{poincaregroup} form the super Poincar\'e algebra. The operators $Z^{AB}$ are known as central charges, and they commute with all elements of the super Poincar\'e algebra. Note that this extension leaves the Poincar\'e algebra as a subalgebra, and so supersymmetry only adds additional structure to the physics we already understand.

There is also an internal symmetry called \emph{R-symmetry} which is the group of transformations on the supercharges that leave the superalgebra invariant. For the case of $\N = 1$, the $R$-symmetry group is U$(1)$ and we can see this in the automorphism
\begin{equation*}
	Q_\alpha \to e^{i \lambda} Q_\alpha, \qquad \bar{Q}_{\dot{\alpha}} \to e^{- i \lambda} \bar{Q}_{\dot{\alpha}}.
\end{equation*}
As the phases for $Q$ and $\bar{Q}$ are opposite, and the only non-trivial commutation relation is for $\{Q, \bar{Q}\}$, we see immediately the algebra will be left unchanged. For the case of $\N > 1$, we can understand the elements of the $R$-symmetry group $S \in G_R$ as those for which
\begin{equation*}
	[Q^A_\alpha, T_i] = \tensor{S}{^A_B} \; Q^B_\alpha, \qquad [\bar{Q}^A_{\dot{\alpha}}, T_i] =-\bar{Q}^B_{\dot{\alpha}} \; \tensor{S}{^A_B} .
\end{equation*}
The general linear transformation $S$ is restricted by the condition that it must commute with the supercharges. When the central charges of the theory vanish, the $R$-symmetry group is simply U$(\N)$. However, when $Z^{AB} \neq 0$, the $R$-symmetry is a subgroup $G_R \subset \text{U}(\N)$ and must be studied on a case-by-case basis.   

Considering the Poincar\'e algebra, we think of the irreducible representations as describing particles. For supersymmetric theories, the particle representations are combined together to form \emph{multiplets} which are populated by \emph{superparticles} which have the same mass, but different spin/helicity. Under action of the supersymmetry algebra, these superparticles transform into each other. The content of the multiplets is dependent on $\N$. As the supercharges commute with the momentum operator and generate fermions from bosons (or vice-versa), we can understand that any bosonic (fermionic) superparticle will be accompanied by at least one fermionic (bosonic) superparticle of the same mass, which we refer to as the \emph{superpartners}. It can be shown that for any supersymmetric theory, the total number of bosons and fermions is the same \cite{Wess:1992cp}. 

While working with quantum field theories, the underlying spacetime is assumed to be flat and the physical system is invariant under global super Poincar\'e symmetries. In this case, we refer to supersymmetry as \emph{global} or \emph{rigid supersymmetry}. If we instead consider a theory of gravity, the super Poincar\'e symmetry is realised as a local (gauge) symmetry, and we call this local supersymmetry, or \emph{supergravity}. 

Supersymmetry transformations on fields are parameterised by a spinor $\epsilon$. In supergravity, the spinor parameter $\epsilon(x)$ itself becomes a spacetime dependent function. The graviton $g_{\mu \nu}$, which is a spin-two massless particle, has $\N$ massless spin-$\tfrac{3}{2}$ gravitino fields $\phi^A_{\mu \alpha}$ as superpartners which are the gauge fields for local supertransformations.

Making the restriction to $\N = 2$, which is the starting point for the solutions in Chapter \ref{ch:planarstu}, the anti-commutation relations of the supercharges simplifies to the form
\begin{equation}
\label{eq:n2algebra}
	\begin{aligned}
	&\{ Q_\alpha^A,  \bar{Q}_{\dot{\beta}}^B \} = 2 \delta^{AB} (\sigma_\mu)_{\alpha \dot{\beta}} P^\mu, \\
	&\{ Q_\alpha^1,Q_\beta^2 \} = -\{ Q_\alpha^2,Q_\beta^1 \} = 2\epsilon_{\alpha \beta} |Z|, \qquad 2|Z| := |Z^{12}|,
	\end{aligned}
\end{equation}
where by making a U(1) phase transformation, it is possible to ensure that $Z^{12}$ is real.

\subsection*{Massive representations}

This discussion follows \cite{Mohaupt:2000gc}. Let us begin with massive representations for $M^2 > 0$, for which we can boost to the rest frame $P_\mu = (-M, 0,0,0)$, such that 
\begin{equation*}
	\sigma^\mu P_\mu = M \sigma_0 = \pmat{M}{0}{0}{M} .
\end{equation*}
We can apply this simplification to \eq{n2algebra} and find that the algebra is written as
\begin{equation*}
	\begin{aligned}
	&\{ Q_\alpha^A,  \bar{Q}_{\dot{\beta}}^B \} = 2 M \delta_{\alpha \dot{\beta}} \delta^{AB} , \\
	&\{ Q_\alpha^1,Q_\beta^2 \} = -\{ Q_\alpha^2,Q_\beta^1 \} = 2\epsilon_{\alpha \beta} |Z|.
	\end{aligned}
\end{equation*}
Next, we take the supercharges and use them to construct the following fermionic operators
\begin{equation*}
	a_\alpha = \frac{1}{\sqrt{2}} \left( Q^1_\alpha + \tensor{\epsilon}{_\alpha^{\dot{\beta}}} \bar{Q}^2_{\dot{\beta}} \right), 
	\qquad
	b_\alpha = \frac{1}{\sqrt{2}} \left(Q^1_\alpha - \tensor{\epsilon}{_\alpha^{\dot{\beta}}} \bar{Q}^2_{\dot{\beta}} \right).
\end{equation*}
Looking at their anti-commutation relations, we find
\begin{equation*}
	\{a_\alpha, \bar{a}_{\dot{\beta}} \} = 2(M + |Z|) \delta_{\alpha \dot{\beta}}, \qquad \{b_\alpha, \bar{b}_{\dot{\beta}} \} = 2(M - |Z|) \delta_{\alpha \dot{\beta}}.
\end{equation*}

We are interested in the irreducible representations of the Poincar\'e algebra, which are complicated to determine as the algebra is not semi-simple. The upshot of Wigner's classification \cite{Wigner:1939cj} is that after setting the mass of the particle (which can be understood as fixing the eigenvalue of Casimir operator built from the momentum operator) there is a remaining symmetry of rotations determined by the so-called \emph{little group}. As such, the irreducible representations of massive particles are represented by the spin $(s)$ of the particle (its representation $[s]$ of the little group SU(2)).\footnote{Considering only bosons, the little group of massive particles is SO(3), but in supersymmetry we necessarily also consider fermions and the inclusion of half-integer spin particles requires working with the double cover of SO(3) which is Spin(3) $\cong$ SU(2).} Setting the spin can be understood as fixing the eigenvalue of the second Casimir operator built from the Pauli–Lubanski operator. 

Setting $M > 0$ and working with the irreducible representations of the little group, we interpret $a_\alpha, \; b_\beta$ as the annihilation operators  
\begin{equation*}
	a_\alpha \ket{s} = 0, \qquad b_\beta \ket{s} = 0,
\end{equation*}
and $\bar{a}_{\dot{\alpha}}, \; \bar{b}_{\dot{\beta}}$ as the creation operators. We can then build a suitable basis of irreducible representations by
\begin{equation*}
	\mathcal{B} = \{ \bar{a}_{\dot{\alpha_1}} \ldots \bar{b}_{\dot{\beta_1}} \ldots \ket{s} \}.
\end{equation*}
Maintaining that our representations are unitary requires that absence of negative norm states. This enforces the bound
\begin{equation*}
	M \geq |Z|,
\end{equation*} 
known as the BPS bound, named after Bogomol'nyi, Prasad and Sommerfield \cite{Bogomolny:1975de, Prasad:1975kr}. States which saturate this bound are known as BPS states, our massive representations fall into two classes. 

When $M > |Z|$, the full algebra is unitary; as we have four creation operators, we have a total of $2^4 \cdot \text{dim}[s]$ states. When the bound is saturated, we find null states which must be removed to maintain unitarity. This can be achieved by setting the operators $b_\beta = \bar{b}_{\dot{\beta}} = 0$ such that the basis of irreducible representations is modified to the form
\begin{equation*}
	\mathcal{B}^\prime = \{ \bar{a}_{\dot{\alpha_1}} \ldots \ket{s} \}.
\end{equation*}
As we have only two creation operators, we have only $2^2 \cdot \text{dim}[s]$ states in the BPS-multiplets.

\subsubsection*{Massless representations}

We now consider massless states, for which irreducible representations of the Poincar\'e group are labelled by their helicity $\lambda$: the quantum number of their representation in the little group SO(2). We can change the helicity through acting with the supercharges $Q^A_{\alpha}, \bar{Q}^A_{\dot{\alpha}}$. As the fermionic operators commute with the momentum operator, we can consider all helicities in a frame in which $P_\mu = (-E,0,0,E)$ such that
\begin{equation*}
	\sigma^\mu P_\mu = E(\sigma_0 + \sigma_3) = \pmat{2E}{0}{0}{0} .
\end{equation*}
Referring back to \eq{n2algebra}, we see that $Q^A_{2} = \bar{Q}^A_{\dot{2}} = 0$, and that $Z^{AB} = 0$, such that the central charges annihilate all states. We have two remaining generators which we normalise as
\begin{equation*}
	a^A = \frac{1}{2\sqrt{E}} Q^A_1, \qquad \bar{a}^A = \frac{1}{2\sqrt{E}} \bar{Q}^A_{\dot{1}},
\end{equation*}
which can be understood as raising or lowering the helicity by one half respectively. With two operators, we can expect $2^2 \cdot \text{dim} [\lambda]$ states per representation. The anti-commutation relations are given by
\begin{equation*}
	\{a^A, \bar{a}^B\} = \delta^{AB}, \qquad \{a^A, a^B\} = \{\bar{a}^A, \bar{a}^B\} = 0.
\end{equation*} 

\begin{table}[!h]
\centering
\def\arraystretch{1}
\begin{tabular}{l|ccc|c}
helicity &   & \multicolumn{1}{c}{state} & & \# states  \\ \hline
 &  &  &  &  \\
$\lambda_{\text{min}}$   &   &  $\ket{\lambda_{\text{min}}}$ & &  1  \\
 &  &  & &  \\
$\lambda_{\text{min}} + \half$     & $\bar{a}^1  \ket{\lambda_{\text{min}}}$ &                           & $\bar{a}^2 \ket{\lambda_{\text{min}}}$ & 2 \\
 &  &  & &  \\
$\lambda_{\text{min}} + 1$ &   & $\bar{a}^1 \bar{a}^2 \ket{\lambda_{\text{min}}}$                         &  & 1  \\
 &  &  &   & 
\end{tabular}
\caption[The four states obtained by acting with raising operators on the Clifford vacuum]{The four states obtained by acting with raising operators on the Clifford vacuum.}
\label{table:fourstates}
\end{table}

To build an irreducible representation, we pick a state of minimal helicity $\ket{E, \lambda_{\text{min}}}$ which is annihilated by all $a^A$. We refer to $\ket{E, \lambda_{\text{min}}} = \ket{\lambda_{\text{min}}}$ as the Clifford vacuum\footnote{We could have in fact used this terminology before. The relations between the fermionic creation and annihilation operators are equivalent to those of a Clifford algebra and the standard Clifford relations can be obtained by taking suitable linear combinations of $a^A$ and $\bar{a}^A$. In this sense, the Clifford algebras can be realised using the fermionic ladder operators. This is discussed in detail in Appendix 5.A of \cite{Green:1987sp}.} \cite{Wess:1992cp} and by acting on it with our raising operators, we can build our states. We summarise the generation of the four possible states into Table \ref{table:fourstates}. We see that for $\N = 2$, we have 4 massless superparticles present in each irreducible representation which split into 2 bosons and 2 fermions. Maintaining that $\lambda_{\text{max}} \leq 2$, we can build all possible base states of the massless representations through picking $\lambda_{\text{min}}$, which is displayed in Table \ref{table:allirreps}. 

\begin{table}[!h]
\centering
\def\arraystretch{1.5}
\begin{tabular}{m{0.7cm}m{0.7cm}m{0.7cm}m{0.7cm}m{0.7cm}m{0.7cm}m{0.7cm}m{0.7cm}m{0.7cm}}
         &                                     &     &               & \multicolumn{3}{c}{$\lambda_{\text{max}}$} &                &      \\
         & \multicolumn{1}{c|}{}               & $2$ & $\frac{3}{2}$ & $1$       & $\frac{1}{2}$       & $0$      & $-\frac{1}{2}$ & $-1$ \\ \cline{2-9} 
         & \multicolumn{1}{c|}{$2$}            & 1   &               &           &                     &          &                &      \\
         & \multicolumn{1}{c|}{$\frac{3}{2}$}  & 2   & 1             &           &                     &          &                &      \\
         & \multicolumn{1}{c|}{$1$}            & 1   & 2             & 1         &                     &          &                &      \\
         & \multicolumn{1}{c|}{$\frac{1}{2}$}  &     & 1             & 2         & 1                   &          &                &      \\
helicity & \multicolumn{1}{c|}{$0$}            &     &               & 1         & 2                   & 1        &                &      \\
         & \multicolumn{1}{c|}{$-\frac{1}{2}$} &     &               &           & 1                   & 2        & 1              &      \\
         & \multicolumn{1}{c|}{$-1$}           &     &               &           &                     & 1        & 2              & 1    \\
         & \multicolumn{1}{c|}{$-\frac{3}{2}$} &     &               &           &                     &          & 1              & 2    \\
         & \multicolumn{1}{c|}{$-2$}           &     &               &           &                     &          &                & 1   
\end{tabular}
\caption[All possible base states of the massless representations of the $\N = 2$ super Poincar\'e algebra]{All possible base states of the massless representations of the $\N = 2$ super Poincar\'e algebra \cite{Wess:1992cp}.}
\label{table:allirreps}
\end{table}

Taken on their own, the multiplets formed through choosing some minimum helicity will not be charge-parity (CP) invariant \cite{Mohaupt:2000gc}. Thus, the physical multiplets we consider will be the pairs of CP conjugated multiplet which then include the corresponding antiparticles. As a concrete example, let us see how this pairing produces the multiplets of $D = 4$, $\N = 2$ supergravity of which we will be primarily concerned with.\footnote{Technically, the hyper multiplets will be set to zero when we solve the equations of motion in Section \ref{sec:stueucinstanton}, but we include them here as we will find that after the dimensional reduction of a four-dimensional theory coupled to vector multiplets, the resulting three-dimensional theory can be described as supergravity coupled to hyper multiplets.}

\begin{itemize}
	\item Supergravity multiplet 
	\begin{equation*}
		(g_{\mu \nu}, \psi_{\mu \alpha}^A, A_\mu)
	\end{equation*}
	The supergravity multiplet is built from combining the massless representations with maximum helicity $2$ and $-1$. This consists of the graviton $g_{\mu \nu}$, two Weyl vector-spinors, called gravitini $\psi^1_\mu, \psi^2_\mu$ and a vector field $A_\mu$ known as the graviphoton. 
	
	\item Vector multiplet
	\begin{equation*}
		(A_\mu, \lambda^A, z)
	\end{equation*}
	The vector multiplet is built from combining the massless representations with maximum helicity $1$ and $0$. It consists of a massless vector field $A_\mu$, two Weyl spinors $\lambda^1, \lambda^2$ and a complex scalar field $z$.  
	
	\item Hyper multiplet 
	\begin{equation*}
		(\gamma^A, q^u)
	\end{equation*}
	A hypermultiplet is built from two representations with maximal helicity-$\half$ and consists of two Weyl spinors $\gamma^1, \gamma^2$ and four real scalar fields $q^1,q^2,q^3,q^4$.
\end{itemize}


\subsection{$\N = 2$ supergravity Lagrangians}
\label{sec:supergravitylag}
We are now ready to study the supergravity Lagrangians which serve as the starting point for the planar solutions we derive in Chapter \ref{ch:planarstu}. In this thesis, we consider only the bosonic field configurations, and hence the fermions in our multiplets are effectively set to zero. The removal of the fermionic matter is a form of \emph{consistent truncation}, which is a removal of some sub-set of the matter content of a theory such that solutions of the truncated theory are also solutions of the full theory. 

Let us first consider the bosonic matter content of pure supergravity. We see that with the gravitini $\psi^A_\mu = 0$, the supergravity multiplet contains the graviton $g_{\mu \nu}$ and a single gauge field $A_\mu$. This is  precisely the same field content as the Einstein-Maxwell theory (see Equation \ref{eq:EMaction}). We can thus interpret the Reissner-Nordstr\"om solution derived in Section \ref{sec:rnsol} as a solution of $\N = 2$ pure supergravity. There is an important distinction to make here though. The Reissner-Nordstr\"om solution may be a solution to supergravity, but we do not say that the Reissner-Nordstr\"om solution is a supersymmetric solution. In Section \ref{sec:supersymmetricblackhole}, we revisit this comment and discuss the extremal Reissner-Nordstr\"om solution as a supersymmetric solution.

The bosonic content of a vector multiplet is made from the vector field $A_\mu$ and a complex scalar field $z$. Here we consider the Lagrangian of $\N = 2$ supergravity coupled to $n_V$ vector multiplets which will have a bosonic field content of the metric, $n_V$ complex scalar fields and $(n_V + 1)$ vector fields. Its Lagrangian is given by \cite{Cremmer:1984hj, Cortes:2009cs}
\begin{equation}
\begin{aligned}
\label{eq:vecmullag}
	 \mathbf{e}_4^{-1} \La &= -\frac{1}{2\kappa_4^2}R - \frac{1}{\kappa^2_4} g_{A\bar{B}} \partial_\mu z^A \partial^\mu \bar{z}^{\bar{B}} + \frac{1}{4} F^I_{\mu \nu} \tilde{G}^{\mu \nu}_{I}, \\
	 &=- \frac{1}{2\kappa_4^2}R - \frac{1}{\kappa_4^2}g_{A\bar{B}} \partial_\mu z^A \partial^\mu \bar{z}^{\bar{B}} + \frac{1}{4} \I_{IJ} F^I_{\mu \nu} F^{J|\mu \nu} + \frac{1}{4} \cR_{IJ} F^I_{\mu \nu} \tilde{F}^{J|\mu \nu}.
\end{aligned}
\end{equation}
where we use the indices $A,B \in \{1,\ldots,n_V\}$ and $I,J \in \{0,\ldots,n_v\}$, the $n_V$ complex scalars are represented by $z^A$, and the $(n_V + 1)$ gauge fields appear in the field strengths $F^I_{\mu \nu}$. $R$ is the Ricci scalar and $\mathbf{e}_4 = \sqrt{-g}$ is the vierbein. The gravitational coupling $\kappa_4^2 = 8\pi G$, and although within a relativity context we often make the choice $G = 1$, it is more common for $\kappa_4^2 = 1$ from a supergravity perspective. In this section, we will keep $\kappa_4$ explicit. We denote the dual field strengths as $G_{I}^{\mu \nu}$, which can be written in terms of the field strength and it's Hodge-dual 
\begin{equation*}
	G_{I}^{\mu \nu} = \cR_{IJ} F^{J | \mu \nu} - \I_{IJ} \tilde{F}^{J | \mu \nu}, 
\end{equation*}
where we use a tilde to represent Hodge-dualisation
\begin{equation*}
	(\star F^I)^{\mu \nu} =: \tilde{F}^{I | \mu \nu} = \half \epsilon_{\mu \nu \rho \sigma} F^{I | \rho \sigma}.
\end{equation*}
Lastly, we have coupling matrices for our matter fields. The scalar field coupling is $g_{A\bar{B}}$, and the gauge field coupling is split into the real and imaginary components: $\N_{IJ} = \cR_{IJ} + i \I_{IJ}$.

The derivation of the Lagrangian \eq{vecmullag} is often found through the process of gauge fixing a theory of superconformal vector multiplets coupled to superconformal supergravity. To properly contextualise some of the following discussion, we offer a sketch of the relationship between these theories but do not offer a full derivation. We refer to \cite{Mohaupt:2000mj, Freedman:2012zz} for a comprehensive overview. 

As we were able to extend the Poincar\'e algebra by including fermionic generators, the same story can be played out with the conformal algebra which itself extends the Poincar\'e algebra by including scale transformations (also known as dilatations) and special conformal transformations. As the superconformal algebra has additional symmetries, to recover Poincar\'e supergravity we should expect to gauge fix the redundant degrees of freedom.

To understand the relationships between these theories, we can consider the multiplets of the superconformal theory. In this discussion we are interested in Weyl multiplet and the superconformal vector multiplets, however, the full discussion requires the inclusion of at least one hyper multiplet too which contributes as a conformal compensator, which we mention again below. The bosonic content of the superconformal vector multiplets is unchanged, containing a vector field $A_\mu$ and a complex scalar $X$. However, the complex scalar fields have the additional symmetry 
\begin{equation*}
	X \to \lambda X, \qquad \lambda \in \mathbb{C}^\star \simeq \mathbb{R}^{> 0} \times \text{U}(1).
\end{equation*}
To obtain the Poincar\'e theory from the superconformal one, we must break both the dilatations generated by $\Real^{> 0}$ and the overall phase transformation from the U(1).

The Weyl multiplet is built from the gauge fields that appear when gauging the rigid theory to obtain superconformal supergravity. Included within the Weyl multiplet are the gauge fields $\tensor{e}{_\mu^a}$, which label the translations and $\tensor{\omega}{_\mu^{ab}}$, which label the Lorentz symmetries. To recover the Poincar\'e theory, we must understand $\tensor{e}{_\mu^a}$ as the vielbein (or the tetrad) and $\tensor{\omega}{_\mu^{ab}}$ as the spin connection. In other words, we require that the local gauge translations of the superconformal theory are equivalent to the diffeomorphisms of the Poincar\'e theory. This is achieved by making the so-called \emph{conventional constraint} relating various gauge fields of the superconformal theory to each other. In particular $\tensor{e}{_\mu^a}$ and $\tensor{\omega}{_\mu^{ab}}$ are no longer independent, but related through the expression for the torsion two-form \cite{Gibbons:lecturenotes}. The conventional constraint requires the matter content of the Weyl multiplet together with two additional multiplets, known as \emph{conformal compensators}. To recover the graviphoton of the supergravity multiplet, we inherit a gauge field by including a superconformal vector multiplet and the second conformal compensator is often taken to be a hyper multiplet.

It can then be shown that Poincar\'e supergravity is gauge equivalent to the local superconformal supergravity theory after imposing the conventional constraint and including the conformal compensators. The upshot is that in order to consider Poincar\'e supergravity coupled to $n_V$ vector multiplets and $n_H$ hyper multiplers, we gauge fix superconformal supergravity coupled to $(n_V + 1)$ superconformal vector multiplets and $(n_H + 1)$ superconformal hyper multiplets. An illustration of this is given in Figure \ref{fig:combination}.

\begin{figure}
\centering
\begin{tikzpicture}[%
    auto,
    block/.style={
      rectangle,
      draw=webgreen,
      thick,
      fill=webgreen!10,
      text width=10em,
      align=center,
      minimum height=3em
    },
    block1/.style={
      rectangle,
      draw=blue,
      thick,
      fill=blue!10,
      text width=10em,
      align=center,
      minimum height=3em
    },
    block2/.style={
      rectangle,
      draw=orange,
      thick,
      fill=orange!10,
      text width=30em,
      align=center,
      minimum height=2.75em
    }
  ]
     \draw[thick, ->] (-5,-0) -- (-5,-3.3);
    \draw[thick, ->] (0,-0) -- (0,-3.3);
    \draw[thick, ->] (5,-0) -- (5,-3.3);
    \draw[thick, ->] (4,-0) -- (1,-3.3);
    \draw (-5,0) node[block] (C) {($n_H + 1$) Hyper multiplets $(q^0, q^u)$};
    \draw (0,0) node[block] (C) {Weyl multiplet $(\tensor{e}{_\mu^a}, \tensor{\omega}{_\mu^{ab}}, \ldots)$};
    \draw (5,0) node[block] (C) {($n_V + 1$) Vector multiplets $(X^I, A^I_\mu)$};
    \draw (-5,-4) node[block1] (C) {$n_H$ Hyper multiplets $(q^u)$};
    \draw (0,-4) node[block1] (C) {Supergravity multiplet $(g_{\mu \nu}, A_\mu) $};
    \draw (5,-4) node[block1] (C) {$n_V$ Vector multiplets $(z^A, A^A_\mu)$};
    \draw (0,-2) node[block2] (C) {Superconformal symmetry breaking};
\end{tikzpicture}
\caption[Diagram of the packaging of the bosonic content of the multiplets of $\N = 2$ Poincar\'e supergravity from the superconformal theory]{Diagram of the packaging of the bosonic content of the multiplets of $\N = 2$ Poincar\'e supergravity from the superconformal theory. By gauge fixing the superconformal $\N = 2$ theory, we obtain the field content of $\N = 2$ Poincar\'e supergravity, with the Weyl multiplet and a single vector multiplet being packaged to give the supergravity multiplet.}
\label{fig:combination}
\end{figure}

Including the additional vector multiplet leads to a miss-match in the number of scalar fields in the superconformal theory: $X^I$, $I \in {0,\ldots,n_V}$ and the super Poincar\'e theory: $z^A$, $A \in {1,\ldots,n_V}$. In fact, viewed from another perspective, one of the benefits with working with the superconformal theory is that the number of gauge fields and scalar fields is balanced. We will explain the utility of this in more detail from the context of finding planar symmetric solutions in Chapter \ref{ch:planarstu}.

Working with the complex fields $X^I$, we must then remember we have spurious degrees of freedom which are set through two gauge-fixing conditions. The first is known as the \emph{dilatation gauge} or D-gauge, which corresponds to setting a scale for the theory, the second is a U(1) symmetry associated to phase transformations of $X^I$. From these complex scalars $X^I$, the physical degrees of freedom can be recovered from the ratios
\begin{equation*}
	z^A = \frac{X^A}{X^0}, \quad \bar{z}^A = \frac{\bar{X}^A}{\bar{X}^0},
\end{equation*}
which are understood to be the scalars of the Poincar\'e supergravity theory after gauge fixing the extra superconformal symmetries. Despite $z^A$ parameterising the physical degrees of freedom of the theory, we instead will work with $X^I$ and keep track of the gauge fixing conditions which we must impose while finding our solutions, where we should understand our Lagrangian as given by
\begin{equation*}
\mathbf{e}_4^{-1} \La = - \frac{1}{2\kappa_4^2}R - \frac{1}{\kappa_4^2}g_{I\bar{J}} \partial_\mu X^I \partial^\mu \bar{X}^{\bar{J}} + \frac{1}{4} \I_{IJ} F^I_{\mu \nu} F^{J|\mu \nu} + \frac{1}{4} \cR_{IJ} F^I_{\mu \nu} \tilde{F}^{J|\mu \nu}.
\end{equation*}
One of the primary benefits is that in this form, the field equations are invariant under symplectic transformations. This is discussed in more detail in Section \ref{sec:electromagneticduality}.

The matrix couplings\footnote{The coupling of the physical scalars $g_{A\bar{B}}$ is distinct to, but can be computed from $g_{I\bar{J}}$. This relation is given by \eq{scalargab} and the calculation is performed explicitly in Appendix \ref{app:stucouplings} for the case of the STU model.} $(g_{I\bar{J}}, \I_{IJ}, \cR_{IJ})$ and hence the dynamics of the Lagrangian, are entirely determined by a holomorphic function known as the prepotential. We denote the prepotential by $F(X^I)$, which is homogenous of degree two. Note that although the same Latin letter is used for the prepotential and the gauge field strength, the presence of the spacetime indices (and context) should help distinguish them. Throughout the thesis, we have been referring to Chapter \ref{ch:planarstu}, stating that we would be finding planar symmetric solutions to the STU model. We can consider the STU model through making a choice for the prepotential \cite{Duff:1995sm}
\begin{equation}
\label{eq:stuprepotential}
	F(X) = \frac{X^1 X^2 X^3}{X^0}.
\end{equation}
The matter content of the STU model is that of $\N = 2$ supergravity coupled to three vector multiplets. As a result, there will be three complex scalar fields (which are sometimes referred to by the letters $s,t,u$) and (3+1) gauge fields. In Appendix \ref{app:stucouplings}, we use this prepotential to find the exact forms of the coupling matrices $g_{A\bar{B}}$ and $\I_{IJ}$ as functions of the physical scalars $z^A$.

We now offer some formulae which help us build an understanding of the couplings in \eq{vecmullag} in terms of the complex scalars $X^I$ and the prepotential $F(X^0, \ldots X^{n_V})$. We will use the notation
\begin{equation*}
	F = F(X^0, \ldots X^{n_V}), \quad F_I = \pardev{F}{X^I}, \quad F_{IJ} = \pardev{F^2}{X^I X^I},
\end{equation*}
for the derivatives of the prepotential.

From the superconformal perspective, the kinetic term for the complex scalars comes with a coupling term $N_{IJ} = -i(F_{IJ} - \bar{F}_{IJ})$, which can be understood as a K\"ahler metric on the target manifold parameterised with our complex holomorphic scalars as coordinates. We can then write the scalar metric as the second derivative of the K\"ahler potential
\begin{equation}
\label{eq:scalarmetric}
    N_{IJ} = \pardev{^2 K}{X^I \partial \bar{X}^{\bar{J}}} \qquad K(X,\bar{X}) = i(X^I \bar{F}_I - F_I X^I).
\end{equation}
We can write the K\"ahler potential in terms of the scalar metric and complex scalars
\begin{equation*}
    \begin{aligned}
            i(X^I \bar{F}_I - F_I X^I) &= i(X^I \bar{F}_{IJ} \bar{X}^{\bar{J}} - X^J F_{IJ} X^I), \\
            &= iX^I\bar{X}^{\bar{J}}( \bar{F}_{IJ} - F_{IJ}),\\
            &= -i^2X^I\bar{X}^{\bar{J}}  N_{IJ} ,\\
            K &= XN\bar{X},
    \end{aligned}
\end{equation*}
where we use the notation $XN\bar{X} = X^I N_{IJ}\bar{X}^{\bar{J}}$. 

The scalar metric $g_{A\bar{B}}$ of the associated Poincar\'e supergravity theory has a related, but different K\"ahler potential given by
\begin{equation}
    \K(X,\bar{X}) = -\log K \quad \Rightarrow \quad e^{-\K} = i(X^I \bar{F}_I - F_I X^I).
\end{equation}
Setting the D-gauge of the theory is equivalent to fixing the value of the potential, commonly the choice is made such that $e^{-\K} = 1$. 

To derive the form of the scalar couplings $g_{A\bar{B}}$, we first take the second derivative of the new potential
\begin{equation}
    g = \pardev{^2 \K}{X^I \bar{X}^{\bar{J}}} dX^I \otimes d\bar{X}^{\bar{J}},
\end{equation}
which is a rank-two tensor field. This is not the metric for the holomorphic coordinates $X^I$ as it has a two-dimensional kernel with null vectors:
\begin{equation*}
	X^I g_{I \bar{J}} = 0, \qquad g_{I \bar{J}} \bar{X}^{\bar{J}} = 0.
\end{equation*}
This means that two real degrees of freedom are non-propagating. As we showed above, we can recover the independent propagating degrees of freedom through considering ratios of the holomorphic coordinates. This allows us to understand this as a metric for the scalars $z^A$, despite being derived from $X^I$. We can explicitly calculate the components of the scalar couplings in terms of the complex scalars $X^I$ and the scalar metric
\begin{equation}
\label{eq:gij}
    g_{I\bar{J}} = - \frac{N_{IJ}}{XN\bar{X}} +  \frac{(N\bar{X})_I(XN)_J}{(XN\bar{X})^2} ,
\end{equation}
where we use the notation such that $(NX)_I = N_{IJ} X^J$. The coupling for the scalar fields is defined from this matrix
\begin{equation}
\label{eq:scalargab}
	 g_{A\bar{B}} =  g_{I\bar{J}} \pardev{X^I}{z^A} \pardev{\bar{X}^{\bar{J}}}{\bar{z}^{\bar{B}}}.
\end{equation}

The couplings of the gauge fields are derived in \cite{Mohaupt:2000mj}, which can be expressed in terms of the prepotential by \cite{Cortes:2009cs}
\begin{equation}
\label{eq:cN}
   \N_{IJ} = \cR_{IJ} + i\I_{IJ} = \bar{F}_{IJ} + i \frac{(XN)_I (XN)_J}{XNX},
\end{equation}
where in our conventions $\I_{IJ}$ is negative-definite. We note here that the field strengths $F^I$ fit into the conformal, but not the Poincar\'e vector multiplets. We choose to work with $F^I$, as $(F^I, G_I)$ is a symplectic vector. We will expand on this in Section \ref{sec:electromagneticduality}. One can obtain the gauge fields of the Poincar\'e supergravity theory from a linear combination of those from the superconformal theory. 

\paragraph{Gauged supergravity}

We take a brief aside to mention the fairly confusing terminology of \emph{gauged supergravity}. We discussed above that to obtain supergravity from rigid supersymmetry, the super Poincar\'e group is gauged making the symmetries local, promoting partial derivatives to covariant derivatives. The new gauge fields are packaged into the supergravity multiplet and from the Lagrangian perspective, we find the appearance of the Einstein-Hilbert term. This (ungauged) supergravity theory has no \emph{additional} gauge symmetries and so the matter is uncharged and all vector fields are Abelian. In contrast, \emph{gauged} supergravity has additional gauge symmetries, which appear in the form of charged matter fields or non-Abelian vector fields. These additional gauge symmetries introduce a scalar potential. In this sense, all supergravity theories are gauged, just some more than others. In this thesis, we will only be interested in solutions to ungauged supergravity theories, but will commonly refer to gauged supergravity when discussing the extension of our work or tangential research.  

\subsection{Supersymmetric black holes}
\label{sec:supersymmetricblackhole}
In Section \ref{sec:supergravitylag}, we mentioned that the bosonic content of $\N = 2$ pure supergravity in four dimensions matched that of Einstein-Maxwell theory, which we studied in Section \ref{sec:rnsol}. As the truncation of fermions is a consistent truncation, we know that the Reissner-Nordstr\"om solution is a solution to supergravity. We now justify the claim that the extremal Reissner-Nordstr\"om solution is supersymmetric following a discussion in \cite{Mohaupt:2000gc}.

Let us first understand what we mean by a supersymmetric solution, and why this is interesting from the perspective of looking at black hole solutions. As an analogy, let us consider the role of Killing vector fields when finding solutions to Einstein's equations. We understand that Killing vector fields generate the symmetries of a spacetime. In Section \ref{sec:maximallysymmetric}, we saw that when we imposed the maximum number of Killing vector fields to be present, the solution to Einstein's equations was almost totally determined. On the flip side, if we assume no symmetry properties and look for some generic solution of Einstein's equations, solving the field equations becomes an incredibly difficult problem. For the black hole solutions we considered, such as the \sch solution and the Reissner-Nordstr\"om solution, we have that they are both static and spherically symmetric and so have four Killing vectors (time-translation and spacial rotations). From the point of view of the field equations, the symmetries which appeared in our ansatz played a role in allowing us to find exact solutions. However, we also can learn about the solutions from the symmetries which are broken. For the \sch solution, spacial translations are broken, but the underlying theory itself is translation invariant. We can then understand all black holes related by some spacelike translation as being equivalent in that they have the same energy \cite{Mohaupt:2000gc}.

There is a supersymmetric analogue to this, where we say a solution is supersymmetric if it is invariant under supersymmetric transformations. We say a field configuration $\Phi_0$ is supersymmetric if
\begin{equation}
\label{eq:killingspinor}
	\delta_{\epsilon(x)} \Phi \big|_{\Phi_0} = 0.
\end{equation} 
The variation is performed on all fields $\Phi$ and evaluated in the field configuration $\Phi_0$. The fermionic parameter $\epsilon(x)$ depends on the spacetime coordinates and is known as a Killing spinor. The Killing spinor equations \eq{killingspinor} are first order equations and are generically easier to solve than the second order Einstein's equations. By imposing that the solution is supersymmetric, we impose restrictions on the field equations which ultimately reduce the complexity of solving Einstein's equations in an analogous way as to how our Killing vector fields reduced the solution space. A maximally supersymmetric solution should have $4\N$ Killing spinors and solutions which break certain supersymmetries have some fractional number of Killing spinors. For spherically symmetric, finite mass solutions of supergravity, we would expect the solutions to be BPS states. Solutions that preserve one-half of the supersymmetries (solutions with $2\N$ Killing spinors) are then said to be $\half$-BPS solutions.

Returning to the extremal Reissner-Nordstr\"om solution, we understand it as embedded into $\N = 2$ pure supergravity and so there is a maximum of eight Killing spinors. As both of the gravitini are truncated out, the supersymmetry variation of the gauge field and graviton are trivial as the background is bosonic. The remaining conditions come from the gravitini variations: 
\begin{equation*}
	\delta_\epsilon \psi^A_\mu \stackrel{!}{=} 0,
\end{equation*}
which are the only variations which impose conditions on the bosonic fields. It can be shown that by inserting in the field content of the extremal solution, there are precisely four Killing spinors. As such, we can understand the extremal Reissner-Nordstr\"om solutions as preserving one-half of the supersymmetries \cite{Mohaupt:2000mj}. Another perspective of seeing the extremal solutions as BPS solutions comes from the analogy of the extremal limit $M = |Q|$ to the BPS limit of $M = |Z|$. We note here that although BPS solutions are always extremal solutions, extremal solutions need not always be BPS solutions. A common first example of this is the Kerr-Newman solution which describes a spinning, charged black hole. Here, we find that the extremal solution is not BPS, and the recovery of a BPS state requires `over rotating' the black hole, producing a spacetime with a naked singularity \cite{Tod:1983pm}.

We can also understand the near-horizon geometry of the extremal solution from a new perspective thanks to supersymmetry. At the asymptotic distance, the spacetime is four-dimensional Minkowski, which has eight Killing spinors for $\N = 2$ supergravity. At the horizon, we find the Bertotti-Robinson solution $AdS_2 \times S^2$, which is a product of maximally symmetric spaces and which also has eight Killing spinors. We then see that the extreme Reissner-Nordstr\"om solution interpolates between two vacua of $\N = 2$ between the asymptotic limit and the horizon. We will see a similar behaviour for the $\half$-BPS state solutions in Section \ref{sec:extbranesol}.

In this thesis, we are concerned with finding non-extremal solutions, and so the Killing spinor equations will not be appropriate for our research. However, in Chapter \ref{ch:planarstu} we will take our four-dimensional solutions and uplift them into higher dimensions. From this higher-dimensional perspective, we will see that we can interpret the extremal limit of our solutions as intersecting brane configurations. In Section \ref{sec:extbranesol} and Section \ref{sec:intersecting}, we will give an overview of the construction of these intersecting brane solutions, which are BPS configurations. Furthermore, in Section \ref{sec:susy6d}, we find that for our class of non-extremal solutions in six dimensions, it is possible to restrict the integration constants in such a way that a BPS solution is recovered, despite the fact we never enforce supersymmetry while solving the equations of motion. 

%************************************
% EM Duality
%************************************
\section{Electromagnetic duality}
\label{sec:electromagneticduality}
 
In this section we take a brief detour to discuss electromagnetic duality and its generalisation, which appears for $\N = 2$ vector multiplets. This will be important to understand when studying the thermodynamic properties of the STU model in Chapter \ref{ch:planarstu}, as we make use of an electric-magnetic duality frame where the magnetic charges $\mathcal{P}^A$, $A=1,2,3$ are replaced by electric charges $\tilde{\mathcal{Q}}_A$. Additionally, find that from the perspective of the duality transformations, we are led to a clear picture of computing the conserved charges of the gauge fields.

\subsection{Maxwell theory}

To explain the idea, we first consider a theory with a single Abelian vector field
$A_\mu$ with field strength $F_{\mu \nu} = \partial_\mu A_\nu - \partial_\nu A_\mu$ and a curved spacetime Maxwell type action\footnote{For a review on the form notation used here, see Appendix \ref{app:conventions}.}
\begin{equation}
\label{eq:maxwell}
S[A] = \int - \frac{1}{2} F \wedge \star F \;.
\end{equation}


We begin our discussion writing down the field equations
\begin{equation*}
	\nabla_\mu F^{\mu \nu} = 0, \qquad \epsilon_{\mu \nu \rho \sigma} \partial^{\nu} F^{\rho \sigma} = 0.
\end{equation*}
The first are the Euler-Lagrange equations which can be found though varying the Lagrangian. The second is the Bianchi identity, which is an integrability condition to ensure the presence of the gauge potential $A_\mu$, and doesn't depend on the spacetime metric. 

The electromagnetic duality is a symmetry of the field equations that interchanges the electric and magnetic fields. The symmetry can be made apparent through writing down the equations of motion in terms of differential forms
\begin{equation}
\label{eq:emdiffform}
	d\star F = 0, \qquad dF = 0.
\end{equation}
Considering the Hodge-star, we can define a dual gauge field by
\begin{equation*}
	\tilde{F} = \star F,
\end{equation*}
replacing $F \to \tilde{F}$ into the equations of motion, we find
\begin{equation*}
	d \star \tilde{F} = d \star^2 F = d F = 0, \qquad d \tilde{F} = d \star F = 0,
\end{equation*}
where we have used that $\star^2 = -1$ for a two-form in four-dimensional Minkowski spacetime. We see that swapping $F \leftrightarrow \tilde{F}$ will leave these two equations invariant, however, their interpretation will change. The Euler-Lagrange equations become the Bianchi identity for the dual field, ensuring that $\tilde{F} = d\tilde{A}$, and the Bianchi identity becomes the new Euler-Lagrange equations for an  action of the form
\begin{equation}
\label{eq:maxwelldual}
S[\tilde{A}] = \int - \frac{1}{2} \tilde{F} \wedge \star \tilde{F} \;.
\end{equation}
The electromagnetic duality is more general than the interchange of $F \leftrightarrow \star F$, and can instead be understood as the invariance of the equations of motion under the mapping from $F$ to some linear combination of $F$ and $\star F$. Another way of thinking about this is that we are free to make some uniform rotation under the action of $\text{Sp}(2, \Real) \simeq \text{SL}(2, \Real)$ of the vector $(F, \star F)$ \cite{Mohaupt:2000mj}.\footnote{From the point of view of the field equations, we can actually further generalise this so the symmetries are generated by GL$(2,\Real)$. Fixing the normalisation of the gauge field reduces this to SL$(2, \Real) \simeq \text{Sp}(2, \Real)$. Once we consider the preservation of the charge quantisation condition, there is a further restriction to $\text{Sp}(2,\mathbb{Z})$.} We note here explicitly that the electromagnetic duality is \emph{not} a symmetry of the Lagrangian. If we make the substitution $F \rightarrow \tilde{F}$ into \eq{maxwell}, we obtain
\begin{equation*}
\begin{aligned}
	S[A] &= \; \int - \half F \wedge \star F, \\
	&\to \int - \half \tilde{F} \wedge \star \tilde{F} = \int - \half \star {F} \wedge \star^2 F = \int \half F \wedge \star F,
	\end{aligned}
\end{equation*}
which we see produces a sign error. The reason that this substitution doesn't work is that $F$ is not a fundamental field; the action \eq{maxwell} is a function of the gauge potential $A$, and not the gauge field $F$. If we want to correctly dualise this action, we must first promote the Bianchi identity to an equation of motion by including it into the action with a Lagrange multiplier. This process is known as \emph{Hodge duality} and in Appendix \ref{app:Hodgeduality}, we offer a general discussion of this process for a $p$-form in a $D$-dimensional spacetime.

We begin by including a three-form $\lambda$ to act as a Lagrange multiplier
\begin{equation}
\label{eq:firstorder}
S[F] = \int -\half F \wedge \star F - d F \wedge \star \lambda.
\end{equation}
This allows the Bianchi identity to become an equation of motion for the Lagrange multiplier
\begin{equation*}
\begin{aligned}
S[ \star \lambda + \star \delta \lambda; F] &= S[ \star \lambda; F] + \int - dF \wedge \star \delta \lambda, \\
0 &= \int - dF \wedge \star \delta \lambda, \quad \Rightarrow \quad dF = 0.
\end{aligned}
\end{equation*}
Varying the action with respect to the field strength, we find a modified equation of motion
\begin{equation*}
\begin{aligned}
		S[ \star \lambda; F + \delta F] &= S[\star \lambda, F] + \int - \delta F \wedge \star  F - d(\delta F)\wedge \star \lambda, \\
		&= S[\star \lambda, F] + \int - \delta F \wedge \star  F + (\delta F)\wedge d \star \lambda, \\
		&= S[\star \lambda, F] + \int \delta F  \wedge \left[ - \star  F + d \star \lambda \right], \\
		&\Rightarrow \star F = d(\star \lambda).
\end{aligned}
\end{equation*}
Collecting these together, we write down the equations of motion for the first order action as
\begin{equation*}
	dF = 0, \qquad \star F = d(\star \lambda), 
\end{equation*}
where we note that the original equations of motion still hold as $d^2 = 0$. In this form, we can perform the dualisation by making the identification
\begin{equation}
	\begin{aligned}
		&\tilde{F} = \star F, \\
		&\tilde{A} = \star \lambda \quad \Rightarrow \quad \tilde{F} = d\tilde{A}  .
	\end{aligned}
\end{equation}
If we substitute these dual fields the action \eq{firstorder}, the resulting action is of the form
\begin{equation}
S[\tilde{A}] =  \int - \half \tilde{F} \wedge \star \tilde{F} \quad \qquad \tilde{F} = d\tilde{A},
\end{equation}
which preserves the duality in the equations of motion, but now also  has the correct sign in the kinetic term when compared to \eq{maxwelldual}. 

The electromagnetic duality also helps us define our conserved charges. Writing our field equations as Bianchi identities:
\begin{equation}
d \tilde{F}=0,\;\;\;dF = 0,
\end{equation}
we can apply Stoke's theorem to obtain values for the electric and magnetic charges\footnote{Note that the normalisation here matches that of the action \eq{maxwell} and will differ in factors from the equations used for the charges of Reissner-Nordstr\"om solution in Section \ref{sec:rnext}, and also the STU model calculations performed in Section \ref{sec:studynamicthermo}. Unfortunately, different models discussed within the thesis have different natural normalisations, so this mismatch is somewhat unavoidable.}
\begin{equation}
\label{charges}
\mathcal{Q} =  \int_X \tilde{F}  =  \int_X  \star F, \qquad
\mathcal{P} =  \int_X F,
\end{equation}
Here we leave the codimension-two manifold generic and understand that for point-like charges the surface $X$ has the topology of a sphere and for solutions with planar symmetry, we take $X$ to be a plane. Note that for planar solutions, the integral over the plane is divergent and so we must instead consider charge densities. The equations of motion and Bianchi identities, which are valid outside charges, tell us that both $F$ and $\tilde{F}$ are closed. This allows one to deform the integration surfaces $X$ continuously, as long as one avoids moving them through the charges. This is not a problem for the solutions we consider, where the charges are located at singularities, which are not included within the domain of our parameters. Often it is convenient to evaluate the charges in a limit where $X$ is evaluated at infinity, and this is in particular how charges are computed throughout this thesis.

The dualisation procedure can be used to replace magnetic charges by electric charges. This can be convenient since in a fixed duality frame electric charges are Noether charges and can couple minimally to the gauge field, whereas magnetic charges are topological and do not have local couplings to the gauge field. For black hole thermodynamics, and in particular, the Euclidean action formalism introduced in Section \ref{sec:eucldeanactionformalism} and used in Chapter \ref{ch:triplewick}, we find it convenient to replace magnetic charges by electric charges. The dual charges are found by replacing $F$ by $\tilde{F}$. Using that 
\[
\tilde{\tilde{F}} =  \star \tilde{F} = \star^2 F = -F,
\]
we find
\begin{eqnarray}
\tilde{\mathcal{Q}}& =&   \int_X \tilde{\tilde{F}} = -  \int_X  F = - \mathcal{P}, \\
\tilde{\mathcal{P}} &=&  \int_X \tilde{F} = \mathcal{Q}.
\end{eqnarray}
and we note that the transformation $(\mathcal{Q}, \mathcal{P}) \rightarrow (-\mathcal{P}, \mathcal{Q})$ is symplectic. 

\paragraph{Symplectic transformations} As a brief review, a symplectic vector space $(V, \omega)$ is built from an even-dimensional vector space $V$ and the symplectic form  $\omega$, which is a non-degenerate, skew-symmetric bilinear form. Picking a basis for $V$, we can write $\omega$ as a matrix, which is commonly picked to be of the form
\begin{equation*}
	\Omega = \pmat{0}{\mathbbm{1}_n}{-\mathbbm{1}_n}{0}.
\end{equation*}
A symplectic transformation is a linear map $L: V \to V$ such that the symplectic form is preserved
\begin{equation*}
	\omega(L u, L v) = \omega(u,v).
\end{equation*}
In a basis, this is the same as the requirement that a matrix $M$ obeys $M^T \Omega M = \Omega$, and a matrix which obeys this relationship is said to be a \emph{symplectic matrix}. The symplectic group $\text{Sp}(2n) \subset \text{GL}(2n)$ is the group of symplectic matrices. Later, when discussing the c-map in Section \ref{sec:cmap}, we will show the real formulation of special geometry allows us to write our $\N = 2$ supergravity theory in a symplectically covariant manner. For more details, we refer to \cite{da2008lectures}. 

\subsection[\titleN$= 2$ vector multiplets]{$\N = 2$ vector multiplets}

Let us now consider the Lagrangian of $n_V + 1$ Abelian gauge fields
\begin{equation}
\label{eq:mulveclag}
	 \mathbf{e}_4^{-1} \La = \frac{1}{4} \I_{IJ} F^I_{\mu \nu} F^{J|\mu \nu} + \frac{1}{4} \cR_{IJ} F^I_{\mu \nu} \star F^{J|\mu \nu}.
\end{equation}
which appears as the gauge field contribution from the $n_V$ vector multiplets in the bosonic field content of $\N = 2$ supergravity \eq{vecmullag}. The coupling matrices $\cR_{IJ}$ and $\I_{IJ}$ depend on the complex scalar fields $X^I$ and are completely determined by the form of the prepotential. 

In order to discuss the underlying symplectic duality, let us first decompose the gauge fields into their selfdual and anti-selfdual components
\begin{equation*}
	F^I_{\mu \nu} = i F^{+ | \; I}_{\mu \nu} + F^{- | \; I}_{\mu \nu}, \qquad \star F^I_{\mu \nu} = -i F^{+ | \; I}_{\mu \nu} + i F^{- | \; I}_{\mu \nu},
\end{equation*}
which are related by
\begin{equation*}
	\star F^{+ | \; I}_{\mu \nu} = - i  F^{+ | \; I}_{\mu \nu}, \qquad \star F^{- | \; I}_{\mu \nu} = i F^{- | \; I}_{\mu \nu},
\end{equation*}
and each piece can be expressed in terms of the gauge field and its Hodge dual
\begin{equation*}
	F^{\pm | \; I}_{\mu \nu} = \half \left( F^I_{\mu \nu} \pm i \star F^I_{\mu \nu} \right).
\end{equation*}
Using the coupling \eq{cN} we can rewrite the above action in the following form \cite{Mohaupt:2000mj}
\begin{equation*}
	\mathbf{e}_4^{-1} \La =  \frac{i}{2} \left( F^{+ | \; I} \N_{IJ} F^{+ | \; J} - F^{- | \; I} \bar{\N}_{IJ} F^{- | \; J}  \right),
\end{equation*}
where we do not include the spacetime indices for cosmetic reasons. 

The field equations of this Lagrangian are given by
\begin{equation*}
	d \left( \N_{IJ} \star F^{+ | \; J} - \bar{\N}_{IJ} \star F^{- | \; J} \right) = 0, \qquad d F^I = 0.
\end{equation*}
Decomposing the Bianchi identities into selfdual and anti-selfdual components, we can write it into the form
\begin{equation*}
	d \left( \star F^{+ | \; I} - \star F^{- | \; I} \right) = 0.
\end{equation*}
As we did in the Maxwell example, we can now define a dual field to make the field equations symmetric 
\begin{equation*}
	G^{\pm}_{I | \mu \nu} = \N_{IJ} F^{\pm | \; J}_{\mu \nu},
\end{equation*}
such that the field equations of the theory are in the form
\begin{equation*}
d \left( \star G^{+ | \; I} - \star G^{- | \; I} \right) = 0	, \qquad d \left( \star F^{+ | \; I} - \star F^{- | \; I} \right) = 0.
\end{equation*}
As with the Maxwell example, the field equations are not simply invariant on the interchange $G^{\pm}_I \leftrightarrow F^{\pm \; |I}$ but for the whole duality rotation \cite{Mohaupt:2000mj}
\begin{equation*}
	\pvec{F^{\pm | \; I} }{G_J^{\pm}} \rightarrow \pmat{\tensor{U}{^I_K}}{\tensor{Z}{^{IL}}}{\tensor{W}{_{JK}}}{\tensor{V}{_J^L}} \pvec{F^{\pm | \; K} }{G_L^{\pm}} = \pvec{\tilde{F}^{\pm | \; I} }{\tilde{G}_J^{\pm}}.
\end{equation*}
From the point of view of the field equations, we can understand the rotation as an element
\begin{equation*}
	\Op = \pmat{\tensor{U}{^I_K}}{\tensor{Z}{^{IL}}}{\tensor{W}{_{JK}}}{\tensor{V}{_J^L}}, \qquad \Op \in \text{GL}(2n_V + 2, \Real).
\end{equation*} 
However, in order to ensure that the Euler-Lagrange equations descend from some Lagrangian after rotation, the group is restricted to $\text{Sp}(2n_V + 2, \Real)$.\footnote{To be more precise, preservation of a Lagrangian formulation reduces from $\text{GL}(2n_V + 2, \Real) \to \text{Sp}(2n_V + 2, \Real) \times \Real^{2n_V + 2}$ and we fix the scalings by setting the normalisation of the gauge fields.} We can realise this restriction by looking at how the gauge coupling transforms under action of $\Op$ and the restrictions we must place to preserve its symmetry. The gauge coupling transforms fractionally linearly \cite{deWit:2001pz}
\begin{equation*}
	\N_{IJ} \to (\tensor{V}{_I^K} \N_{KL} + W_{IL}) \tensor{[(U + Z \N)^{-1}]}{^L_J}.
\end{equation*} 
Imposing that $\N_{IJ} = \N_{JI}$ restricts the sub-matrices
\begin{equation*}
	\begin{aligned}
		U^T W- W^T U &= 0, \qquad U^T V - W^T Z = \mathbbm{1}_n, \\
		Z^T V - V^T Z &= 0,
	\end{aligned}
\end{equation*} 
which are precisely the relations which define an element of the symplectic group.

We note that this transformation acts continuously, but once charge quantisation is taken into account it is broken to a discrete subgroup $\text{Sp}(2n_V + 2, \mathbb{Z})$. As the vector is covariant under symplectic transformations, we say that $(F^{\pm | \; I} G_J^{\pm})^T$ is a \emph{symplectic vector}. Within the context of $\N = 2$ supergravity, there is another symplectic vector of the form $(X^I, F_I)^T$, where to be precise, we remind the reader $F_I = \partial_I F$ is the derivative of the prepotential. This duality for the complex scalar and (the derivative of) the prepotential leads to the equivalence of various theories built from prepotentials of different forms \cite{deWit:2001pz}. For more details on how various pieces of the $\N = 2$ Lagrangian transform under the action of the symplectic group, we refer to \cite{Mohaupt:2008gt}. A totally comprehensive review of these duality rotations within the context of $\N = 2$ supergravity can be found in \cite{LopesCardoso:2019mlj}.

Let us now consider this duality with a concrete example which we will use in later calculations for the thermodynamics in Chapter \ref{ch:triplewick}. In the planar solutions of the STU model we derive in Chapter \ref{ch:planarstu}, we make restrictions on the field configurations to allow for exact solutions to the differential equations. As a consequence, we find that the coupling matrices are restricted by $\mathcal{R}_{IJ} = 0$ and the remaining coupling matrix $\mathcal{I}_{IJ}$ is diagonal. Let us now consider the duality discussed above in this restricted case. Under these assumptions, the Lagrangian \eq{mulveclag} simplifies into the form
\begin{equation}
\label{eq:vecmullagsimple}
\mathbf{e}_4^{-1} \La =  
 + \frac{1}{4} \left(  \I_{00} F^0_{\mu \nu} F^{0|\mu \nu} 
+ \sum_{A=1}^3  \I_{AA} F^A_{\mu \nu} F^{A|\mu \nu} \right) + \cdots,
\end{equation}
where we remind the reader that the STU model is built from $n_V = 3$ vector multiplets. This amounts to ($3+1$) copies of the type of vector field Lagrangian we have considered in the Maxwell example, but now the gauge field appears with a coupling which is spacetime dependent.

The symplectic transformation in this special case becomes an exchange of the electric and magnetic fields while additionally inverting the coupling. To demonstrate this, let us return to the single Abelian vector field example 
\begin{equation}
\label{eq:maxwellgen}
S[A] = \int - \frac{1}{2g^2} F \wedge \star F \;,
\end{equation}
but now we include an explicit coupling $g$ which is allowed to depend on the spacetime coordinates. Terms like this commonly appear in the study of black holes in supergravity, where the gauge field is coupled to a dilaton such that the coupling is $g^2 = e^{\alpha \phi}$ \cite{Horne:1992bi, Horowitz:1992jp}. For our case when considering the $\N = 2$ vector multiplets, we would take $g^{-2} = -\I_{II}$, where the additional minus sign appears as $\I_{IJ}$ is negative-definite. The field equations of \eq{maxwellgen} take the form
\begin{equation}
	d \left( \frac{1}{g^2} \star F \right) = 0, \qquad d F = 0.
\end{equation}
As such, we are prompted to define the dualised gauge fields in the following way:
\begin{equation*}
	\tilde{F} = \frac{1}{g^2}  \star F.
\end{equation*}
Substituting $F$ as a function of $\tilde{F}$ into the field equations with the relationship
\begin{equation*}
	F = - g^2 \star \tilde{F},
\end{equation*}
gives the following set of field equations for the dual field
\begin{equation*}
	\begin{aligned}
		d \left( \frac{1}{g^2} \star F \right) &= - d \left( \star^2 \tilde{F} \right) = d \tilde{F} = 0,\\
		d F &= d \left(- g^2 \star \tilde{F} \right) = 0.
	\end{aligned}
\end{equation*}
We can make these equations appear more symmetric with the original pair though introducing a dual coupling $g^{-1} = \tilde{g}$ such that the field equations can be written as
\begin{equation*}
	dF = 0 \;\; \Leftrightarrow \;\; d \left( \frac{1}{\tilde{g}^2} \star \tilde{F} \right), \qquad	d \left( \frac{1}{g^2} \star F \right)  = 0 \;\; \Leftrightarrow \;\; d\tilde{F} = 0 .
\end{equation*}
We can thus understand this dualisation procedure as exchanging the Euler-Lagrange equations and the Bianchi identity with the additional constraint that the gauge coupling is inverted. For the dual field, the Bianchi identity ensures that $\tilde{F} = d\tilde{A}$ and the new Euler-Lagrange equations exist as the equations of motion for a dual action
\begin{equation}
\label{eq:maxwelldualgen}
S[\tilde{A}] = \int - \frac{1}{2 \tilde{g}^2} \tilde{F} \wedge \star \tilde{F} \;.
\end{equation}
The process of dualising the action \eq{maxwellgen} into \eq{maxwelldualgen} follows the steps outlined for the Maxwell theory, where one must promote the Bianchi identity to an equation of motion through introducing a Lagrange multiplier. This is carried out in full detail in Appendix \ref{app:Hodgeduality} for a $p$-form kinetic term in D dimensions. Applying this procedure to \eq{vecmullagsimple}, where it is understood that $F_A$ are magnetically charged, we can write down the Hodge dualised action that is purely electric
\begin{equation*}
 e_4^{-1} \La =  
 + \frac{1}{4 \kappa_4^2} \left(  \I_{00} F^0_{\mu \nu} F^{0|\mu \nu} 
+ \sum_{A=1}^3  \I^{AA} \tilde{F}_{A | \mu \nu} \tilde{F}_A^{\mu \nu} \right) + \cdots,
\end{equation*}
where we draw attention to the inversion of the couplings $\I_{AA} \rightarrow \I^{AA}$.

Again, considering the Bianchi identities of the gauge field and its dual, we can compute the conserved charges. The electric and magnetic charges are given by
\begin{equation}
\label{eq:charges}
\mathcal{Q} =  \int_X \tilde{F}  = \int_X \frac{1}{g^2} \star F \;, \qquad
\mathcal{P} =  \int_X F \;.
\end{equation}
where we now see that the electric charge appears with the coupling. Returning to our concrete example, before dualisation we have the electric and magnetic charges: $(\mathcal{Q}_0, \mathcal{P}^A)$. We can map these to be purely electric charges: $(\mathcal{Q}_0, \tilde{\mathcal{Q}}_A)$, where $\tilde{\mathcal{Q}}_A=-\mathcal{P}^A$ with the above duality map. We can compute the value of the conserved charges with the following integrals
\begin{eqnarray}
\mathcal{Q}_0 &=&  \int_X \tilde{F}_0 =  \int_X 
\star \left( - \mathcal{I}_{00}  F^0\right)\;, \\
\tilde{\mathcal{Q}}_A &=& - \int_X F^A =
 \int_X \star \left( - \mathcal{I}^{AA} \tilde{F}_A \right) \;.
\end{eqnarray}


%************************************
% DIM. RED
%************************************
\section{Dimensional reduction}
\label{sec:dimreduction}
In this section, we review the procedure of dimensional reduction. We will use dimensional reduction as a tool at several points within this thesis, using it to better understand the structure of our supergravity solutions, and as a method to simplify the equations of motion while finding solutions to gravitational systems. In Section \ref{sec:kkred}, we give a brief discussion of the history of dimensional reduction to allow a physical perspective of the subsequent calculations. In Section \ref{sec:dimredcalc}, we give a prescription of how to dimensionally reduce a generic $(D+1)$-dimensional Lagrangian, which is suitable for the application of the various models we consider. Finally, in Section \ref{sec:studimred} we give a worked example of the dimensional reduction of the STU model from six to four dimensions, following \cite{Chow:2013gba}. Later, while discussing the c-map in Section \ref{sec:cmap}, we will see a second example of dimensional reduction as we compactify $\N = 2$ supergravity coupled to $n_V$ vector multiplets from four to three dimensions, following \cite{Mohaupt:2011aa, Vaughan:2012}.

\subsection{Kaluza-Klein reduction}
\label{sec:kkred}
Dimensional reduction first appeared when Kaluza \cite{Kaluza:1921tu} suggested to Einstein that one could unify electromagnetism with gravity by considering a five-dimensional spacetime. In essence, Kaluza's suggestion was to study a five-dimensional metric where the electromagnetic gauge potential appears as components of the metric. By ensuring that this metric also contained a four-dimensional subspace that described the physical spacetime we experience, Kaluza aimed to geometrically capture both electromagnetism and gravity within a single tensor. Years later, Klein noticed that if this extra fifth dimension was small and compact, this five-dimensional spacetime effectively reduces to four-dimensional gravity coupled to electromagnetism \cite{Klein:1926fj}. 

In essence, the Kaluza-Klein reduction assumes a cylindrical condition, where the higher-dimensional spacetime is assumed to have the form $M_5 = S^1 \times M_4$, with one compact coordinate wrapped into a tight circle
\begin{equation*}
	x^0 \simeq x^0 + 2\pi L.
\end{equation*}
This periodicity allows us to interpret this compactification coordinate as an angular coordinate, and we understand $L$ as the radius of circle we compactify over.

For the following discussion, we will assume we are working with a spacetime of dimension $(D+1)$, with coordinates $x^{\hat{\mu}}$, where $\hat{\mu} = \{0, \ldots, D\}$, and our compact coordinate is $x^0$. Let us study how this identification effects a scalar field $\Phi(x^{\hat{\mu}})$. We can make the Fourier expansion for the field
\begin{equation*}
	\Phi(x^{\hat{\mu}}) = \Phi(x^0, x^\mu) = \sum_{n} \phi_n (x^\mu) e^{inx^0 / L}.
\end{equation*}
In this form, we find the so-called Kaluza-Klein tower of an infinite number of $D$-dimensional fields $\phi_n(x^\mu)$, with masses $\frac{|n|}{L}$. Klein's `cylinder condition' is to truncate out every mode except the massless mode $\phi_0$, by assuming that the radius of the circle is very very small. This truncation happens as when the radius shrinks, the masses of the massive modes grow. As the massive modes then exist at energy levels far above the effective field theories we compactify over, they are safely ignored. 
It is this procedure of keeping only the massless modes in a Fourier expansion which we define as the \emph{Kaluza-Klein reduction}. In Section \ref{sec:dimredcalc}, this procedure is carried out for a general Lagrangian containing the Einstein-Hilbert and gauge field contribution.

Reducing pure gravity from five to four dimensions produces a Lagrangian of the following form
\begin{equation*}
	S = \int d^4 x \sqrt{-g} \left( -\half R - \half \partial_\mu \phi \partial^\mu \phi - \frac{1}{4} e^{\sqrt{3} \phi} F_{\mu \nu} F^{\mu \nu} \right),
\end{equation*}
which seems to be an incredible result (see Section \ref{sec:dimredcalc} for the reduction formula). Both gravity and electromagnetism come falling out of a five dimensional theory, with the field strength $F = dV$ behaving as we would expect a U(1) gauge field to \cite{Errington:2016}. Although mathematically elegant, the physical motivation of describing gravity and electromagnetism together using a fifth, compact dimension was questioned. Firstly, although the field content is the same, the equations of motion in five dimensions include the further constraint \cite{Blau}
\begin{equation*}
	\hat{R}_{00} = 0 \quad \Rightarrow \quad F_{\mu \nu} F^{\mu \nu} = 0.
\end{equation*}
It is only after setting this that we obtain the equations of motion we saw in Equation \eq{einstein} through considering $R_{0 \mu} = 0$ and $R_{\mu \nu} = 0$. We cannot say Einstein-Maxwell theory is a consistent truncation of five-dimensional pure gravity without ignoring part of the five-dimensional equations of motion. We must also ask where the motivation for this additional dimension comes from. The cylindrical condition imposes that our physical fields are independent of this extra coordinate, and so it appears in Kaluza-Klein theory as a spectator or mathematical oddity.

These criticisms resulted in Kaluza-Klein compactification not being considered seriously as a unification tool or a description of `real-world' physics. However, years later within the context of higher-dimensional supergravity and the various superstring theories in ten dimensions, dimensional compactification came back into focus. We will discuss this with more context in Section \ref{sec:pbranes}.

In our work, we will not be concerned with phenomenological questions, but instead we will use dimensional compactification as a tool to solve equations of motion in Chapter \ref{ch:planarstu} and as a way to understand our four-dimensional solutions as brane configuration in Chapter \ref{ch:brane}.  The fact remains though, that if we hope to see a stringy understanding of quantum gravity, somewhere along the way we'll need to find a consistent way to tie up all the extra dimensions.

As a closing remark, in this thesis we will only be concerned with reductions over the circle $S^1$ or tori $T^n$ (which is $n$ repeated reductions over $S^1$). However, the compact, internal manifold can have a much richer structure. We will not be employing any of these generalisations, but include suitable references for further reading. Performing a Kaluza-Klein reduction over the sphere, one obtains a reduced theory containing Yang-Mills gauge fields \cite{Popekk}. In the context of supergravity, compact manifolds of special holonomy are reduced over to break supersymmetry, see for example Calabi-Yau reductions \cite{Candelas:1990pi, Bodner:1990zm, Duff:1994zt} which produce $\N = 2$ supergravity solutions in four dimensions, reducing the total number of supercharges by three-quarters. Rather than compactifying over manifolds, another method to break supersymmetry comes from reducing over orbifolds \cite{Bailin:1999nk}, or orientifolds \cite{Bianchi:1990yu}. The Scherk-Schwarz reduction \cite{Scherk:1978ta} is a generalisation of the Kaluza-Klein reduction where the assumption that the lower-dimensional fields are independent of the internal coordinates is relaxed. One application for the Scherk-Schwarz reduction is to produced gauged supergravity in lower dimensions \cite{Hull:2002wg, Inverso:2017lrz}.

\subsection{Dimensional reduction formulae}
\label{sec:dimredcalc}
In this section, we give the result of the Kaluza-Klein reduction of a generic model containing contributions from an Einstein-Hilbert term and a gauge field term. The results will be stated without proof, and are included as a reference for when we perform reductions within the thesis. A wealth of good references exist with explicit computations such as \cite{Vaughan:2012, Errington:2016, Popekk, Dempster:2014}. 

The first step in the reduction is to make the Kaluza-Klein ansatz for the metric tensor, suitable for the reduction over the circle
\begin{equation}
\label{eq:KKansatz}
  ds^2_{D+1} = - \epsilon e^{2 \beta \phi} (dx^0 + V_\mu dx^\mu)^2 + e^{-2\alpha \phi} ds_D^2,
\end{equation}
where $x_0$ is the compact direction, $\phi$ is the Kaluza-Klein scalar and $V_\mu$ is the Kaluza-Klein vector. The constants, $\alpha$ and $\beta$, are set by requiring that the Einstein-Hilbert part of the action reduces in the so-called `Einstein-frame', in which the Ricci scalar has no coupling with the Kaluza-Klein scalar, and the kinetic term for the Kaluza-Klein scalar has a factor of $-\tfrac{1}{2}$.\footnote{A common technique is to set $\alpha = 0$ and $\beta = 1$ to simplify the equations during the reduction. The Einstein-frame can then be recovered after a conformal transformation. An alternative choice is called the `string frame', where $\alpha, \; \beta$ are picked so the Ricci scalar has a factor of $e^{-2\phi}$.} This is done through picking
\begin{equation*}
\alpha^2 = \frac{1}{2(D-1)(D-2)}, \quad \beta = (D-2)\alpha .
\end{equation*}
The constant $\epsilon$ is used to allow us to express the reduction over $x^0$ of arbitrary signature, where
\begin{equation*}
\epsilon = \begin{cases}
	-1 \text{ if $x^0$ is spacelike}, \\
	\hspace{7pt} 1 \text{ if $x^0$ is timelike},
\end{cases}  
\end{equation*}
such that the signature of the $(D+1)$-dimensional metric is
\begin{equation*}
	\{ - \epsilon \underbrace{- \ldots -}_{t-\text{times}} + \ldots + \}.
\end{equation*}
This allows us to consider the Kaluza-Klein reduction over both timelike and spacelike dimensions. Generally, the coordinate is assumed to be spacelike and we reduce from one Minkowski theory to another. However, reducing over timelike coordinates and considering the resulting Euclidean theory has a wealth of applications too. One that we focus on in particular is the timelike reduction of $\N = 2$ supergravity from four to three dimensions in the context of the c-map, which is discussed properly in Section \ref{sec:cmap}, and used in Chapter \ref{ch:planarstu} to obtain non-extremal, planar symmetric solutions to the STU model. More generally, for a review of special geometry and Euclidean supergravity see \cite{Cortes:2015wca, Cortes:2009cs, Cortes:2003zd, Cortes:2005uq}.

In the following, we will denote $(D+1)$-dimensional fields with a hat, and the $(D+1)$-dimensional indices count from $\hat{\mu} = \{0, \ldots, D \}$, whereas the unhatted will not include the compact dimension: $\mu = \{1, \ldots, D \}$. When written in form notation, we will use the Hodge-star operator, which is also dimension-dependent; however, we do not include hats and instead understand the Hodge operator from context.

We begin with a generic $(D + 1)$-dimensional Lagrangian containing the Einstein-Hilbert term, and a gauge contribution from a $p$-form
\begin{equation*}
	\hat{S} = \hat{S}_{EH} + \hat{S}_{F^2} = \int d^{D+1} x \sqrt{-\hat{g}} \left[ - \half \hat{R} - \frac{1}{2 p!} \hat{F}_{\hat{\mu}_1 \ldots \hat{\mu}_p}\; \hat{F}^{\hat{\mu}_1 \ldots \hat{\mu}_p } \right].
\end{equation*}
Using the following identities
\begin{equation*}
 \star A \wedge B = \frac{1}{p!} A_{\mu_1 \ldots \mu_p} B^{\mu_1 \ldots \mu_p} \star 1, \qquad  \star 1 = \sqrt{-\hat{g}} d^{D+1} x,
\end{equation*}
for some $p$-forms $A, B$, we can write Lagrangian using differential forms
\begin{equation*}
	\hat{S} = \int_{\M_{D+1}}  -\star \frac{1}{2} \hat{R} - \frac{1}{2} \hat{F}_p \wedge \star \hat{F}_p,
\end{equation*}
for more discussion on form notation, see Appendix \ref{app:conventions}. Note that the $p$-form is assumed to be exact: $F = dA_{p-1}$ for some gauge potential $A_{p-1}$. 

Commonly we consider Lagrangians which also contain a scalar kinetic term, however, the Kaluza-Klein reduction of this term is trivial. We replace a derivative over $(D+1)$ dimensions with the derivative over the remaining $D$ dimensions
\begin{equation*}
	\partial_{\hat{\mu}} \varphi \partial^{\hat{\mu}} \varphi \rightarrow \partial_{\mu} \varphi \partial^{\mu} \varphi.
\end{equation*}
Note that in form notation, this is $d\varphi \wedge \star d\varphi \rightarrow d\varphi \wedge \star d\varphi$.

We can break the reduction down into two parts. Starting with the Einstein-Hilbert term, it is found that the Ricci scalar reduces as
\begin{equation}
\label{eq:dimeh}
\begin{aligned}
	\hat{S}_{EH} &= \int_{\M_{D+1}} -\half \star \hat{R}  \\
	&\to \int_{\M_D} \left[ - \half \star R - \half d \phi \wedge \star d\phi + \half \epsilon e^{\left( \frac{2D-2}{D-2} \right)^\half \phi} dV \wedge \star dV \right].
\end{aligned}
\end{equation}
We see that from the $(D+1)$-dimensional Ricci scalar, the resulting $D$-dimensional spacetime has contributions from a $D$-dimensional Ricci scalar, as well as a scalar kinetic term from the Kaluza-Klein scalar and a gauge field contribution from the two-form $dV$, due to the Kaluza-Klein vector.

The reduction of a $p$-form gauge field contribution is given by
\begin{equation*}
\begin{aligned}
	\hat{S}_{F^2} &=   \int_{\M_{D+1}} -\half F_p \wedge \star F_p  \\
	&\to \int_{\M_D}  e^{(2p - D)\alpha \phi} \left[-e^{-(\beta + 2\alpha)\phi} \half F_{p-1} \wedge \star F_{p-1} - \epsilon e^{\beta \phi} \half (F_p - V \wedge F_{p-1}) \wedge \star (F_p - V \wedge F_{p-1}) \right],
\end{aligned}
\end{equation*}
where we do not substitute in the values for $\alpha, \beta$ to make this easier on the eye. We can simplify this slightly, by defining a new gauge potential
\begin{equation*}
	B_{p-1} = A_{p-1} - V \wedge A_{p-2}, \qquad G_{p} = dB_{p-1},
\end{equation*}
allowing us to write down the Kaluza-Klein reduction of a $p$-form as
\begin{equation}
\label{eq:dimgauge}
	S_{F_p^2} = \int_{\M_D} \left[- \epsilon e^{(2p - 2)\alpha \phi} \half G_{p} \wedge \star G_p - e^{2 (p - D) \alpha \phi} \half F_{p-1} \wedge \star F_{p-1} \right].
\end{equation}
In short, the Kaluza-Klein reduction of a $p$-form in $(D+1)$ dimensions produces an action in $D$ dimensions including both a $p$-form and a $(p-1)$-form. In the special case where we reduce a two-form gauge field with potential $\hat{A}_{\hat{\mu}}$, we obtain a gauge field term for $A_\mu$, and a scalar kinetic term for $A_0$.

One-piece we do not consider here in full generality is the dimensional reduction of topological terms arising from the gauge fields. The form of the topological term (and hence its reduction) is dependent on the values of both $p$ and $D$ and so is hard to discuss generically. However, we will reduce a four-dimensional topological term built from a two-form in Section \ref{sec:cmap}, which we now outline.

When $(D+1)$ is even and the gauge field present is a $(p+1)$-form, and $p+1 = \frac{D+1}{2}$ is also even, there can be a topological term
\begin{equation*}
	\hat{S}_{\text{top}} = \int \half d \hat{A}_p \wedge d \hat{A}_p .
\end{equation*}
Decomposing the gauge field $d\hat{A}_{p}$, we can write it as 
\begin{equation*}
	d\hat{A}_{p} = dA_p + dx^0 \wedge dA_{p-1}.
\end{equation*}
Expanding this relation in the topological term 
\begin{equation*}
\begin{aligned}
		\hat{S}_{\text{top}} &= \int \half (dA_p + dx^0 \wedge dA_{p-1}) \wedge (dA_p + dx^0 \wedge dA_{p-1}), \\
		&= \half \int dA_p \wedge dx^0 \wedge dA_{p-1} + dx^0 \wedge dA_{p-1} \wedge dA_p ,\\
		&= \half \int (-)^{p+1} dx^0 \wedge dA_p \wedge dA_{p-1} + (-)^{p(p+1)}dx^0 \wedge dA_p \wedge dA_{p-1},\\
		&= \int dx^0 \wedge dA_p \wedge dA_{p-1}.
\end{aligned}
\end{equation*}
Integrating we obtain the reduced term in $D$ dimensions
\begin{equation*}
	S_{\text{top}} = \int dA_p \wedge dA_{p-1}.
\end{equation*}
The case we consider in the c-map in Section \ref{sec:cmap} is when $D= 3$ and $p = 1$.

Written with Lorentz indices expanded, we can collect the Ricci scalar and gauge kinetic term computations to write a $D$-dimensional action after reduction. Here we make the choice $p = 2$, as this will be the most common choice throughout the thesis
\begin{equation*}
\begin{aligned}
	S = \int d^D x \sqrt{-g} \; &\bigg[ \;- \half R - \half \partial_\mu \phi \partial^\mu \phi + \frac{1}{4} \epsilon e^{\left( \frac{2D-2}{D-2} \right)^\half \phi} V_{\mu \nu}   V^{\mu \nu} \\
	&+ \epsilon e^{\left( \frac{2(n-2)}{n-1} \right)^\half \phi} \half \partial_\mu \chi \partial^\mu \chi - \frac{1}{4} e^{\left( \frac{2}{(D-2)(D-1)}\right)^\half \phi} G_{\mu \nu} G^{\mu \nu} \bigg].
\end{aligned}
\end{equation*}
where $V_{\mu \nu}$ is the field strength from the Kaluza-Klein vector, and $\chi = A_0$ is a scalar field, where we decompose the gauge potential as $\hat{A} = \chi dx^0 + A_\mu dx^\mu$. The explicit form of the new gauge field is
\begin{equation*}
	G_{\mu \nu} = F_{\mu \nu} - 2V_{[\mu} \partial_{\nu]} \chi.
\end{equation*}

\subsection[Dimensional reduction of \titleN$=1$, $6D$ supergravity]{Dimensional reduction of $\N=1$, $6D$ supergravity}
\label{sec:studimred}

As a concrete example of Kaluza-Klein compactification, we now perform the reduction of $\N=1$, six-dimensional supergravity coupled to tensor multiplets over $T^2 = S^1 \times S^1$ \cite{Smith_1985, Duff:1995sm}. This reduction produces a four-dimensional theory with the $\N = 2$ supergravity multiplet, and three vector multiplets. As discussed, this is the STU model which is the focus of Chapter \ref{ch:planarstu}.

\subsubsection{Reduction from six to five dimensions}

The action for $\N=1$, six-dimensional supergravity coupled to one tensor multiplet is \cite{Duff:1995sm}
\begin{equation}
\label{eq:6dlag}
    \hat{S}_{6D} =\int d^6x \sqrt{-\hat{g}} \left(-\hat{R} - \frac{1}{2}  \partial^{\hat{\mu}} \phi \partial_{\hat{\nu}} \phi -\frac{1}{12} e^{-\sqrt{2}\phi} H_{\hat{\mu} \hat{\nu} \hat{\rho}} \hat{H}^{\hat{\mu} \hat{\nu} \hat{\rho}}\right).
\end{equation}
Where $\phi$ is the dilaton, and $\hat{H}_{\hat{\mu} \hat{\nu} \hat{\rho}}$ is a three-form field strength $\hat{H} = d\hat{B}$. To perform the reduction, we make the ansatz
\begin{equation}
\label{eq:KKansatz6d}
  ds^2_{6} =  e^{2 \beta \sigma} (dz_6 + \tilde{\mathbb{A}}_1)^2 + e^{-2\alpha \sigma} ds_5^2,
\end{equation}
where we reduce over the compact, spacelike coordinate $z_6$, the Kaluza-Klein scalar is represented by $\sigma$, to differentiate it from the dilaton appearing in \eq{6dlag}, and the Kaluza-Klein vector is labeled $\tilde{\mathbb{A}}_1 = (\tilde{\mathbb{A}}_1)_\mu dx^\mu$. 

Referring to the above equations, setting $D = 5$ in \eq{dimeh}, the Ricci scalar and dilaton terms reduce to
\begin{equation*}
	S_{EH + \phi^2} = \int d^5x \sqrt{-g} \left(-R - \half \partial_\mu \sigma \partial^\mu \sigma - \frac{1}{4} e^{\frac{4}{\sqrt{6}} \sigma} (\tilde{\mathbb{F}}_{1})^2 - \half  \partial_\mu \phi \partial^\mu \phi  \right),
\end{equation*}
where we use $\tilde{\mathbb{F}}_1 = d\tilde{\mathbb{A}}_1$, and we will use $F^2 = F_{\mu \nu}F^{\mu \nu}$ throughout this reduction to reduce the number of indices in our expressions. 

The gauge term can be computed with \eq{dimgauge}, setting $D=5$ and $p = 3$ to find
\begin{equation*}
	S_{H^2} = \int d^5x \sqrt{-g} e^{-\sqrt{2}\phi} \left(-\frac{1}{12}e^{-\frac{2}{\sqrt{6}}\sigma} \mathbb{H}^2 -\frac{1}{4}e^{\frac{2}{\sqrt{6}}\sigma} (\mathbb{F}_2)^2 \right),
\end{equation*}
where we denote the $(p-1)$-form $\tilde{\mathbb{F}}_2 = d\tilde{\mathbb{A}}_2$, and the $p$-form as $\mathbb{H} = H - \tilde{\mathbb{F}}_2 \wedge \tilde{\mathbb{A}}_1$. Before reducing this five-dimensional solution, we can do some book-keeping to simplify the above expression. As we are in five dimensions, a three-form is Hodge-dual to a two-form. Performing the Hodge dualisation explained in Section \ref{sec:electromagneticduality}, we can find that
\begin{equation*}
	\tilde{\mathbb{F}}_3 = e^{- \frac{2}{\sqrt{6}} \sigma - \sqrt{\phi}} \mathbb{H}.
\end{equation*} 
Inserting this into the above action we find we can write
\begin{equation*}
	S_{H^2} = \int d^5x \sqrt{-g} \left(-\frac{1}{4}e^{\frac{2}{\sqrt{6}}\sigma+\sqrt{2}\phi} (\tilde{\mathbb{F}}_3)^2 -\frac{1}{4}e^{\frac{2}{\sqrt{6}}\sigma-\sqrt{2}\phi} (\mathbb{F}_2)^2 \right)
\end{equation*}
We can further simplify our terms by reparameterising the scalar fields $\phi, \sigma$, defining three new scalar fields
\begin{equation*}
    h_1 = e^{2 \sigma/\sqrt{6}}, \qquad  h_2 = e^{\phi/\sqrt{2} -  \sigma/\sqrt{6}}, \qquad h_3 = e^{-\phi/\sqrt{2} -  \sigma/\sqrt{6}},
\end{equation*}
which are subject to the constraint $h_1 h_2 h_3 = 1$, maintaining the degrees of freedom captured by the scalar fields. We compute the derivatives of the scalar fields as
\begin{equation*}
        \partial_\mu h_1 = \frac{2 \partial_\mu \sigma}{\sqrt{6}} h_1, \qquad \partial_\mu h_2 = \left(\frac{\partial_\mu \phi}{\sqrt{2}} - \frac{ \partial_\mu \sigma}{\sqrt{6}}\right)h_2, \qquad \partial_\mu h_3 = \left(-\frac{\partial_\mu \phi}{\sqrt{2}} - \frac{ \partial_\mu \sigma}{\sqrt{6}}\right)h_3.
\end{equation*}
Squaring and summing, we see that we can write the kinetic term for the scalar fields $\phi, \sigma$ in terms of $h_i$:
\begin{equation*}
       (\partial \phi)^2 + (\partial \sigma)^2  = \sum_{i=1}^3  (h_i)^{-2} (\partial h_i)^2,
\end{equation*}
which in form notation is given by
\begin{equation}
	\star d\phi \wedge d\phi + \star d\sigma \wedge d\sigma =  \sum_{i=1}^3 \frac{\star dh^i \wedge dh^i}{(h^i)^{2}}.
\end{equation}
Written using the scalars $h_i$, the five-dimensional action simplifies into the form
\begin{equation*}
\begin{aligned}
	S_{5D} &= \int d^5x \sqrt{-g} \left[-R - \sum_{i = 1}^3 \frac{1}{2 h_i^2} \left( \partial_\mu h_i \partial^\mu h_i + \half (\tilde{\mathbb{F}}_i)^2 \right) \right] ,\\
	&= 	\int_{M_5} -\star R  - \half \sum_{i = 1}^3 \frac{1}{h_i^2} \left(  dh_i \wedge \star dh_i + \tilde{\mathbb{F}}_i \wedge \star \tilde{\mathbb{F}}_i \right) .
\end{aligned}
\end{equation*} 
As a consistency check, we can compare this action with the one derived in \cite{Cortes:2009cs} and see that the only difference is an overall scaling of the gauge fields
\begin{equation}
	\mathbb{F}_i = \frac{1}{\sqrt{3}} \F_i \qquad \mathbb{H} = -\frac{1}{\sqrt{3} (h^3)^2 } \star_5 \F_3 ,
\end{equation}
where we use $\F_i$ for the notation of the gauge fields appearing in Equation (6.2) of \cite{Cortes:2009cs}.

 \subsubsection{Reduction from five to four dimensions}

We are now in a position to reduce the five-dimensional action to four dimensions. We re-write the action, where we now include hats on the five-dimensional fields, ready to reduce to four dimensions
\begin{equation*}
\begin{aligned}
	\hat{S}_{5D} = \int &d^5x \sqrt{-\hat{g}} \; \bigg[ -\hat{R}
	- \frac{1}{2}  \partial^{\hat{\mu}} \phi \partial_{\hat{\nu}} \phi
	- \frac{1}{2}  \partial^{\hat{\mu}} \sigma \partial_{\hat{\nu}} \sigma \\ 
	&- \frac{1}{4} e^{-4 \sigma / \sqrt{6}}  (\hat{\tilde{\mathbb{F}}}_1)^2  + \frac{1}{4} e^{-2\phi/\sqrt{2} +  2\sigma/\sqrt{6}}(\hat{\tilde{\mathbb{F}}}_2)^2  + \frac{1}{4} e^{2\phi/\sqrt{2} +  2\sigma/\sqrt{6}}(\hat{\tilde{\mathbb{F}}}_3)^2  \bigg].
\end{aligned}
\end{equation*}
We begin through making the ansatz
\begin{equation}
\label{eq:KKansatz5d}
  ds^2_{4} =  e^{2 \beta \lambda} (dz_5 + \tilde{A}_4)^2 + e^{-2\alpha \lambda} ds_4^2,
\end{equation}
where we are reducing over a spacelike coordinate $dz_5$, the Kaluza-Klein scalar is denoted with a $\lambda$ to keep it distinct from the previous ones, and the Kaluza-Klein vector is denoted $\tilde{A}_4$. 

Using \eq{dimeh} with $D = 4$, we find that the Einstein-Hilbert reduces to
\begin{equation*}
	S_{EH} = 
	\int d^4x \sqrt{-g} \left(-R - \half \partial_\mu \lambda \partial^\mu \lambda - \frac{1}{4} e^{\sqrt{3} \lambda} \tilde{F}_4 \right),
\end{equation*}
where we write the field strength of the Kaluza-Klein vector as $\tilde{F}_4 = d\tilde{A}_4$. We can consider all three gauge field reductions simultaneously, and using \eq{dimgauge}, with $D = 4$ and $p = 2$ we find that
\begin{equation*}
	\hat{\tilde{\mathbb{F}}}_i \wedge \star \hat{\tilde{\mathbb{F}}}_i \rightarrow e^{\frac{\lambda}{\sqrt{3}}}  \tilde{F}_i \wedge \star \tilde{F}_i + e^{-\frac{2\lambda}{\sqrt{3}}}  d \chi_i \wedge \star d \chi_i, \qquad \tilde{F}_i = d\tilde{A}_i,
\end{equation*}
where the five-dimensional vectors $\mathbb{A}_i$ have been reduced to the pair of four-dimensional vectors $\tilde{A}_i$ and a scalars $\chi_i$. 

Before we combine all these terms together, it is helpful to make the following redefinition of the scalars
\begin{equation*}
	 \varphi_1 = -\frac{2}{\sqrt{6}} \sigma - \frac{1}{\sqrt{3}} \lambda , \quad \varphi_2 = -\frac{1}{\sqrt{2}} \phi + \frac{1}{\sqrt{6}} \sigma - \frac{1}{\sqrt{3}} \lambda , \quad \varphi_3 = \frac{1}{\sqrt{2}} \phi + \frac{1}{\sqrt{6}} \sigma - \frac{1}{\sqrt{3}} \lambda ,
\end{equation*}
which, after doing so allows us to write the Lagrangian in the following form:
\begin{equation*}
	S_{4D} = \int d^4x \sqrt{-g} \left[ R - \half (d \varphi_i \wedge \star d\varphi_i + e^{2\varphi_i} d\chi_i \wedge \star d\chi_i) - \half e^{-\varphi_1 - \varphi_2 - \varphi_3} \left(\tilde{F}_4 \wedge \star \tilde{F}_0 + e^{2\varphi_i} \tilde{F}_i \wedge \star \tilde{F}_i\right) \right].
\end{equation*}
We will return to this reduction process in Chapter \ref{sec:upliftstu} when we perform the uplift of a four-dimensional solution to five, and then six dimensions. During these calculations, we will work under an additional assumption which we call the `purely imaginary condition'. This is equivalent to the condition $\chi_i = 0$,\footnote{These scalar fields are sometimes referred to as axions, and the `purely imaginary condition', is sometimes also known as the `axion free condition'} and so under these conditions, the four-dimensional action further simplifies into the form
\begin{equation}
	S_{4D} = \int d^4x \sqrt{-g} \left( -R - \half (d \varphi_i \wedge \star d\varphi_i) - \half e^{-\varphi_1 - \varphi_2 - \varphi_3} \left(\tilde{F}_4 \wedge \star \tilde{F}_4 + e^{2\varphi_i} \tilde{F}_i \wedge \star \tilde{F}_i\right) \right).
\end{equation}

\subsection*{A note on transgression terms}

Before we conclude this section, we make a note on an omission we have made during our calculations. In the Hodge dualisation of the three-form in five dimensions, an additional topological term appears in the five-dimensional action:
\begin{equation}
\label{eq:5dtopological}
	S_{\text{top}} = \int d^5x \; \tilde{\mathbb{F}}_2 \wedge \tilde{\mathbb{F}}_3 \wedge \tilde{\mathbb{A}}_1.
\end{equation}
The appearance of topological terms after Hodge dualisation is explained in the general case in Appendix \ref{app:Hodgeduality}. When performing the dimensional reduction from five to four dimensions, if the topological term \eq{5dtopological} is also reduced, the gauge fields in the four-dimensional theory are modified by the so-called transgression terms. For the reduction of a $p$-form, all transgression terms are proportional to the resulting $(p-1)$-form.

For this current reduction, the modification of the gauge field terms is given precisely in \cite{Chong:2004na, Chow:2014cca}, where they also include the resulting four-dimensional topological term
\begin{equation*}
	S_{\text{top}} = \int d^4x \; \chi _1 \left(F_1 \wedge F_4 + F_2 \wedge F_3 \right),
\end{equation*}
which also arises from the reduction of \eq{5dtopological}. The reason we do not include the topological terms in the computations of this section is that we are interested in the results of this calculation in the application of Chapter \ref{ch:planarstu}, where we demand that $\chi_i = 0$. As all the modifications to the two-forms in four dimensions are proportional to $\chi_i$ (as well as the four-dimensional topological term itself), we effectively set to zero all contributions from the dimensional reduction of the topological term. Thus, we find that it is sufficient to ignore them in favour of computational clarity in this example.

\section{The c-map}
\label{sec:cmap}

In this section, we use a Kaluza-Klein reduction to dimensionally reduce $\N = 2$ supergravity coupled to $n_V$ vector multiplets in four dimensions, producing a theory described by $n_H = n_V + 1$ hypermultiplets coupled to $\N = 2$ supergravity in three dimensions \cite{Ferrara:1989ik}. This mapping between the target spaces of the four and three-dimensional theories is known as the \emph{c-map}. In particular, we use that in three dimensions it is possible to dualise vector fields into scalar fields, leading to a simplification of the equations of motion for the solutions considered in Section \ref{sec:stueucinstanton}. 

We follow the work of \cite{Mohaupt:2011aa, Cortes:2015wca}, which developed the c-map through introducing a new formulation in terms of special real coordinates and the corresponding Hesse potential; the development of which led to a series of publications \cite{Dempster:2015, Mohaupt:2011aa, Errington:2014bta, Dempster:2016} on the construction of non-extremal, stationary solutions in theories of four-dimensional ${\cal N}=2$ vector multiplets coupled to gauged and ungauged supergravity. It was this work that we continued in \cite{Gutowski:2019iyo} which led to the cosmological solutions of $\N = 2$ supergravity we present in Chapter \ref{ch:planarstu}.

\subsection{Dimensional reduction}

We begin with the four-dimensional Lagrangian for $\N=2$ supergravity coupled to $n_V$ vector multiplets, described in Section \ref{sec:supergravity}. We note that this Lagrangian can be obtained from the reduction of type II theories over a Calabi-Yau manifold \cite{Bodner:1990zm}, or heterotic theories compactified over $K3 \times T^2$ \cite{Duff:1994zt, Antoniadis:1995vz}. We rewrite the Lagrangian with our four-dimensional field content, which will be denoted by a hat, as will the spacetime indices: $\hat{\mu} \in \{ 0, 1,2,3 \}$.
\begin{equation}
\begin{aligned}
\label{eq:cmap4dlag}
 \mathbf{e}_4^{-1} \La_4 = -\frac{1}{2}\hat{R}_4 - g_{A\bar{B}} (z,\bar{z}) \partial_{\hat{\mu}} z^A \partial^{\hat{\mu}} \bar{z}^{\bar{B}} + \frac{1}{4} \I_{IJ} (z,\bar{z}) \hat{F}_{\hat{\mu}\hat{\nu}}^I \hat{F}^{J|\hat{\mu}\hat{\nu}} + \frac{1}{4} \cR_{IJ}(z,\bar{z}) \hat{F}_{\hat{\mu}\hat{\nu}}^I \tilde{\hat{F}}^{J|\hat{\mu}\hat{\nu}} .
\end{aligned}
\end{equation}
Note that, in our conventions, $g_{A\bar{B}}$ is positive-definite and $\I_{IJ}$ is negative-definite. 

Rather than working with the scalar fields $z^A$, we choose to work with special coordinates $X^I$, which introduces symplectic covariance of the field equations. As explained in Section \ref{sec:supergravitylag}, the coordinates $X^I$ live in a larger ambient space which introduces gauge freedoms that will need to be fixed. To return back to the physical hypersurface, parameterised by $z^A$, we must fix both the dilatations and the U(1) phase transformations \cite{Errington:2014bta}.

The dilatations can be broken through imposing the D-gauge
\begin{equation}
\label{eq:dgauge1}
-i(X^I \bar{F}_I - F_I\bar{X}^I) = 1.
\end{equation}
Fixing the U(1) phase transformation necessarily breaks the symplectic covariance, and so we choose to leave U(1) symmetry of the special coordinates until we have solved the equations of motion. For more details on this we refer to \cite{Mohaupt:2011aa, Vaughan:2012}. 

In Chapter \ref{ch:planarstu}, the gauge condition is fixed while imposing the `purely imaginary' field configuration, which are a set of conditions for the physical fields. The presence of the U(1) gauge symmetry means that when counting the independent degrees of freedom, it is important that we realise that one of the relations within the set is not a condition of physical fields, but rather a gauge fixing. Any condition which is not U(1) invariant would be suitable, however, conventionally and within this thesis we identify the condition
\begin{equation*}
	\text{Im}(X^0) = 0,
\end{equation*}
as the gauge fixing relation. The `purely imaginary' condition will be discussed in more detail in Section \ref{sec:restricted} from the context of finding solutions to the equations of motion.

The map from the generic (inhomogenous) coordinates $z^A$ to homogenous coordinates $X^I$ in the Lagrangian is induced by the following transformation
\begin{equation}
\bar{g}_{A\bar{B}} (z,\bar{z}) \partial_{\hat{\mu}} z^A \partial^{\hat{\mu}} \bar{z}^{\bar{B}} \rightarrow g_{I\bar{J}} (X,\bar{X}) \partial_{\hat{\mu}} X^I \partial^{\hat{\mu}} \bar{X}^{\bar{J}}.
\end{equation}
where we have used that $g_{I\bar{J}}$ has a two-dimensional kernel, so that the number of independent degrees of freedom remains the same.\footnote{The two-dimensional kernel descending from the scale and U(1) invariance of the superconformal theory.} Rewriting the Lagrangian \eq{cmap4dlag} in terms of the special coordinates
\begin{equation}
\mathbf{e}_4^{-1} \La_4 = -\frac{1}{2}\hat{R}_4 - g_{I\bar{J}} (X,\bar{X}) \partial_{\hat{\mu}} X^I \partial^{\hat{\mu}} \bar{X}^{\bar{J}} + \frac{1}{4} \I_{IJ} \hat{F}_{\hat{\mu}\hat{\nu}}^I \hat{F}^{J|\hat{\mu}\hat{\nu}} + \frac{1}{4} \cR_{IJ} \hat{F}_{\hat{\mu}\hat{\nu}}^I \tilde{\hat{F}}^{J|\hat{\mu}\hat{\nu}}.
\end{equation}

Although we will only need the reduction over a timelike circle in Section \ref{sec:dimred}, in this section, we follow the work of \cite{Mohaupt:2011aa} and allow for the dimensional reduction of \eq{cmap4dlag} over either a timelike or spacelike direction. We do this as explained in Section \ref{sec:dimredcalc}, by introducing factors of $\epsilon$. 

We impose an ansatz on our four-dimensional solution such that the metric can be decomposed into the appropriate form to allow the standard Kaluza-Klein dimensional reduction
\begin{equation}
\label{eq:KKant}
ds_4^2 = -\epsilon e^\phi (dy + V_\mu dx^\mu)^2 + e^{-\phi} ds_3^2.
\end{equation}
As we are in $(D + 1) = 4$ dimensions, $\alpha = \beta = \half$, and $\phi, V_\mu$ are the Kaluza-Klein scalar and vector respectively. 

Under this assumption, the gauge fields $\hat{A}^I_{\hat{\mu}}$ decompose into
\begin{equation*}
	\hat{A}^I = \zeta^I dy + (A_\mu^I - \zeta^I V_\mu ) dx^\mu,
\end{equation*}
where $\hat{A}^I_{y} = \zeta^I$, and we include $\zeta^I V_\mu$ to eliminate `naked' vector potentials, allowing us to write everything in terms of gauge invariant field strengths \cite{Cortes:2015wca}.

Performing the reduction of \eq{cmap4dlag} using the rules described in Section \ref{sec:dimredcalc}, the three-dimensional Lagrangian is given by
\begin{equation}
\begin{aligned}
 \mathbf{e}_3^{-1} \La_3 &= \frac{1}{2}\left(-R_3 - \frac{1}{2}\partial_\mu \phi \partial^\mu \phi + \frac{1}{4}\epsilon e^{2\phi}V^{\mu\nu} V_{\mu\nu} \right) - g_{I\bar{J}} \partial_\mu X^I \partial^\mu \bar{X}^{\bar{J}} \\
 &+ \frac{1}{4}e^\phi \I_{IJ}(F^I_{\mu\nu} + \zeta^IV_{\mu\nu})(F^{J|\mu\nu} + \zeta^JV^{\mu\nu}) \\
 &- \frac{1}{2}\epsilon e^{-\phi} \I_{IJ} \partial_\mu\zeta^I \partial^\mu \zeta^J - \frac{1}{2} \epsilon \es^{\mu\nu\rho} \cR_{IJ}(F^I_{\mu\nu} + \zeta^IV_{\mu\nu})\partial_\rho\zeta^J .
\end{aligned}
\end{equation}
The first line comes from the reduction of the Einstein-Hilbert and scalar kinetic terms, the remainder of the Lagrangian comes from the reduction of the vector fields. 

A great benefit of working in three dimensions, is that we are able to simplify the Lagrangian through the dualisation of the vector fields $(A^I, V)$ into scalar fields $(\tilde{\zeta}_I, \tilde{\phi})$. We can write down the dual Lagrangian by including a Lagrange multiplier term, given by
\begin{equation}
\mathbf{e}_3^{-1} \La_{\text{Lm}} = \frac{1}{2} \epsilon \varepsilon^{\mu\nu\rho} \big(F^I_{\mu\nu}\partial_\rho\tilde{\zeta}_I - V_{\mu\nu}
\partial_\rho \big(\tilde{\phi} - \frac{1}{2} \zeta^I\tilde{\zeta}_I \big)\big).
\end{equation}
We can then eliminate the vector terms through computing their equations of motion from the new Lagrangian $\tilde{\La}_3 = \La_3 + \La_{\text{Lm}}$. The resulting computation allows us to write the vector terms as functions of the scalars
\begin{equation}
\begin{aligned}
 V_{\mu\nu} &= 2e^{-2\phi}\varepsilon_{\mu\nu\rho} \big( \partial^\rho \tilde{\phi} + \frac{1}{2}(\zeta^I \partial^\rho \tilde{\zeta}_I - \tilde{\zeta}_I \partial^\rho \zeta^I) \big),\\
 F_{\mu \nu}^I &= \lambda e^{-\phi} \I^{IJ} \epsilon \varepsilon_{\mu\nu\rho} \left[\partial^\rho \tilde{\zeta}_J - \cR_{JK} \partial^\rho \zeta^K \right] - \zeta^I V_{\mu \nu}.
\end{aligned}
\end{equation}
These can be substituted back into $\tilde{\La}_{3}$ to obtain
\begin{equation}
\label{eq:3dunscaled}
\begin{aligned}
 \La_3 &=- \half R_3 - \frac{1}{4}\partial_\mu \phi \partial^\mu \phi - g_{I\bar{J}} \partial_\mu X^I \partial^\mu \bar{X}^{\bar{J}} \\
 &- e^{-2\phi} \left[ \partial^\rho \tilde{\phi} + \half \left(\zeta^I \partial^\rho \tilde{\zeta}_I - \tilde{\zeta}_I \partial^\rho \zeta^I \right) \right] \left[ \partial_\rho \tilde{\phi} + \half \left(\zeta^I \partial_\rho \tilde{\zeta}_I - \tilde{\zeta}_I \partial_\rho \zeta^I \right) \right] \\
 &- \frac{\epsilon}{2} e^{-\phi} \left[\I_{IJ} \partial_\mu\zeta^I \partial^\mu \zeta^J + \I^{IJ} 
	\left(\partial^\rho \tilde{\zeta}_I - \cR_{IM} \partial^\rho \zeta^M \right) 
	\left(\partial_\rho \tilde{\zeta}_J - \cR_{JN} \partial_\rho \zeta^N \right) \right],
\end{aligned}
\end{equation}
where we have dropped the tilde over $\La_3$ as we will no longer work with the previous Lagrangian. We relegate some additional computational details into Appendix \ref{app:cmapfurther}.

This Lagrangian is of the standard form often found in the literature \cite{Ferrara:1989ik}, however, in \cite{Mohaupt:2011aa} it is explained that inspired by the r-map\footnote{The r-map can be understood in analogy to the c-map, where instead a five-dimensional supergravity theory coupled to $n_v$ vector multiplets is dimensionally reduced to four dimensions.} \cite{Cortes:2009cs}, it is favourable to perform a field redefinition through scaling the complex coordinates
\begin{equation}
Y^I := e^{\phi/2} X^I.
\end{equation}
In these new variables, the D-gauge becomes an expression relating the Kaluza-Klein scalar to our new scalar fields
\begin{equation}
-i(Y^I \bar{F}_{\bar{I}} - F_I\bar{Y}^{\bar{I}}) = e^\phi.
\end{equation}
The matrix $g_{I\bar{J}}$ is homogenous of degree two and so transforms as
\begin{equation}
g_{I\bar{J}}(X,\bar{X}) = e^\phi g_{I\bar{J}}(Y,\bar{Y}).
\end{equation}
The coupling matrices $\cR_{IJ}$ and $\I_{IJ}$ are homogenous of degree zero and so
\begin{equation}
\I_{IJ}(X,\bar{X}) = \I_{IJ}(Y,\bar{Y}), \quad \cR_{IJ}(X,\bar{X}) = \cR_{IJ}(Y,\bar{Y}).
\end{equation}
With all of this taken into account, the Lagrangian \eq{3dunscaled} is now of the form
\begin{equation}
\label{eq:c4dl}
\begin{aligned}
 \mathbf{e}_3^{-1} \La_3 &= - \frac{1}{2}R_3 - g_{I\bar{J}}(Y,\bar{Y}) \partial_\mu Y^I \partial^\mu \bar{Y}^J - \frac{1}{4} \partial_\mu \phi \partial^\mu \phi \\
 &- e^{-2\phi} \left[ \partial^\rho \tilde{\phi} + \half \left(\zeta^I \partial^\rho \tilde{\zeta}_I - \tilde{\zeta}_I \partial^\rho \zeta^I \right) \right] \left[ \partial_\rho \tilde{\phi} + \half \left(\zeta^I \partial_\rho \tilde{\zeta}_I - \tilde{\zeta}_I \partial_\rho \zeta^I \right) \right] \\
 &- \frac{ \epsilon}{2} e^{-\phi} \left[\I_{IJ} \partial_\mu\zeta^I \partial^\mu \zeta^J + \I^{IJ} 
	\left(\partial^\rho \tilde{\zeta}_I - \cR_{IM} \partial^\rho \zeta^M \right) 
	\left(\partial_\rho \tilde{\zeta}_J - \cR_{JN} \partial_\rho \zeta^N \right) \right].
\end{aligned}
\end{equation}

\subsection{Special real formulation}
We will now write the Lagrangian \eq{c4dl} using special real coordinates. These calculations follow \cite{Mohaupt:2011aa}. We decompose the complex coordinate $Y^I$ and the derivative of the prepotential as
\begin{equation}
\begin{aligned}
 Y^I &= x^I + iu^I(x,y), \\
 F_I &= y_I + iv_I(x,y),
\end{aligned}
\end{equation}
and collect both $x^I$ and $y_I$ to form the special real coordinates
\begin{equation}
q^a := \pvec{x^I}{y_I}.
\end{equation}
With the real scalars $q^a$, we can write the \emph{Hessian metric}
\begin{equation*}
	H_{ab} = \pardev{^2 H}{q^a \partial q^b},
\end{equation*}
where $H(q^a)$ is the \emph{Hesse potential} and we can relate the Hesse metric to the K\"ahler metric by the relationship
\begin{equation*}
	g = \text{Re} \left(N_{IJ} dY^I \otimes d\bar{Y}^{\bar{J}} \right) = H_{ab} q^a \otimes q^b.
\end{equation*}
The Hesse potential itself is related to the holomorphic prepotential by a Legendre transformation replacing $u^I$ with $y_I$ as independent variables \cite{VicenteCortes2002}:
\begin{equation*}
\begin{aligned}
\label{eq:hessepotential}
	H(x,y) &= 2 \text{Im} F(x,y) - 2 y_I u^I(x,y), \\
	&= \frac{i}{2} \left(Y^I \bar{F}_I - F_I \bar{Y}^I \right) = - \half \phi.
\end{aligned}
\end{equation*}
The D-gauge written in terms of the rescaled real scalar fields takes the simple form of
\begin{equation*}
	-2H(q^a) = e^\phi.
\end{equation*}

We now briefly discuss two additional metric tensors which appear in the following computations. In Section \ref{sec:supergravitylag} we wrote down an additional K\"ahler potential by taking the logarithm of the K\"ahler potential of the corresponding superconformal theory. In a similar way, we define a new Hesse potential by
\begin{equation*}
	\tilde{H} := - \half \log (-2H), \qquad \tilde{H}_{ab} := \pardev{\tilde{H}}{q^a \partial q^b}.
\end{equation*}
The tensor $\tilde{H}_{ab}$ is a non-degenerate rank two tensor field and as such can be interpreted as a new Hessian metric. The inverse metric $\tilde{H}^{ab}$ has a corresponding Hesse potential with a flipped sign:
\begin{equation*}
	\tilde{H}^{ab} := \pardev{(-\tilde{H})}{q_a \partial q_b},
\end{equation*}
where the dual coordinate $q_a$ has been defined as:
\begin{equation}
 q_a := \tilde{H}_a := \pardev{\tilde{H}}{q^a} = -\frac{H_a}{2H}= - \frac{1}{H}\pvec{v_I}{-u^I}.
\end{equation}
This definition implies the following relations
\begin{equation}
 q_a = -\tilde{H}_{ab}q^b \Rightarrow q^a = - \tilde{H}^{ab}q_b, \qquad \partial_\mu q_a = \tilde{H}_{ab} \partial_\mu q^b,
\end{equation}
where we have used that $\tilde{H}_a$ is homogeneous of degree -1 and then an application of the chain rule in the second relation. We can use $\tilde{H}_{ab}$ to lower the metric index in $\partial_\mu \hat{q}^a$ to obtain a covector field
\begin{equation}
 \partial_\mu \hat{q}_a := \tilde{H}_{ab} \partial \hat{q}^b.
\end{equation}

The second additional metric comes from considering the vector field coupling. We can express the vector kinetic matrix $\N_{IJ}$ in terms of a new, real matrix $\hat{H}_{ab}$. In \cite{Cortes:2011aj} its form is found to be
\begin{equation}
\hat{H}_{ab} := \pmat{\I + \cR\I^{-1}\cR}{-\cR\I^{-1}}{-\I^{-1}\cR}{\I^{-1}}.
\end{equation}
This matrix appears as the coupling matrix of the vector kinetic terms after dimensional reduction, and we find that this particular expression simplifies our work and is preferable to rewriting $\cR_{IJ}$ and $\I_{IJ}$ themselves.

These three metrics can be related through the expressions which are proved in \cite{Mohaupt:2011aa}
\begin{equation}
\label{eq:hattohes}
\begin{aligned}
\tilde{H}_{ab} &=  -\frac{1}{2H} \left(H_{ab} - \frac{H_aH_b}{H} \right), \\
               &= \frac{1}{H}\hat{H}_{ab} - \frac{2}{H^2} (\Omega_{ac}q^c)(\Omega_{bd}q^d).
\end{aligned}
\end{equation}

We now turn to writing the Lagrangian \eq{c4dl} in terms of our special real coordinates. We outline some steps and refer to \cite{Mohaupt:2011aa, Vaughan:2012} for an in-depth discussion. The Kaluza-Klein scalar can be written in terms of the Hesse potential
\begin{equation}
e^\phi = -i(Y^I \bar{F}_{\bar{I}} - F_I \bar{Y}^{\bar{I}}) = -2H,
\end{equation}
and the scalar kinetic term is found to be \cite{Mohaupt:2011aa}
\begin{equation}
\begin{aligned}
\label{eq:LA1}
g_{I\bar{J}} dY^I d\bar{Y}^J &= \left[-\frac{1}{2H}H_{ab} + \frac{1}{4H^2}H_aH_b + \frac{1}{H^2}(\Omega_{ac}q^c)(\Omega_{bd}q^d) \right]dq^a dq^b .
\end{aligned}
\end{equation}
Using that
\begin{equation}
\phi = \log (-2H) \Rightarrow \pardev{\phi}{H} = \frac{1}{H},
\end{equation}
and using the chain rule
\begin{equation}
\partial_\mu \phi = \pardev{\phi}{H} \pardev{H}{q^a}\partial_\mu q^a = \frac{1}{H} H_a \partial_\mu q^a,
\end{equation}
we can rewrite the kinetic term for the Kaluza-Klein scalar field as
\begin{equation}
\label{eq:LA2}
\frac{1}{4} \partial_\mu \phi\partial^\mu \phi = \left[ \frac{1}{4H^2}H_aH_b \right] \partial_\mu q^a \partial^\mu q^b.
\end{equation}

We are now left to rewrite the terms arising from the gauge fields. We can do this elegantly by defining the scalar field
\begin{equation}
\hat{q}^a := \frac{1}{2}\pvec{\zeta^I}{\tilde{\zeta}_I},
\end{equation}
which can be related to the physical field strengths by
\begin{equation}
 \partial_\mu \zeta^I = \hat{F}^I_{\mu 0}, \qquad \partial_\mu \tilde{\zeta}_I = \hat{G}_{I|\mu 0}.
\end{equation}
With a bit of algebraic work, we are able to rewrite gauge field terms as
\begin{equation}
\begin{aligned}
 &-\frac{\epsilon }{2} e^{-\phi} \bigg[\I_{IJ} \partial_\mu \zeta^I \partial^\mu \zeta^J + \I^{IJ} \big(\partial_\mu \tilde{\zeta}_I - \cR_{IK}\partial_\mu \zeta^K \big) \big(\partial^\mu \tilde{\zeta}_J - \cR_{JL}\partial^\mu \zeta^L \big)\bigg] \\
 &= \frac{ \epsilon}{2} \hat{H}_{ab} \partial_\mu \hat{q}^a \partial^\mu \hat{q}^b.
\end{aligned}
\end{equation}
We can then relate the above to the Hesse potential using the relation \eq{hattohes}. When this is done we obtain
\begin{equation}
\label{eq:LA3}
\frac{ \epsilon}{H}\hat{H}_{ab} \partial_\mu \hat{q}^a \partial^\mu \hat{q}^b =  \epsilon \left[ -\frac{1}{2H} H_{ab} + \frac{1}{2H^2} H_aH_b + \frac{2}{H^2} (\Omega_{ac}q^c)(\Omega_{bd}q^d) \right] \partial_\mu \hat{q}^a \partial^\mu \hat{q}^b .
\end{equation}
Piecing together equations \eq{LA1}, \eq{LA2} and \eq{LA3} we can write down the three-dimensional Lagrangian in terms of the special real coordinates and the Hesse potential
\begin{equation}
\begin{aligned}
 \mathbf{e}_3^{-1} \La_3 &= - \frac{1}{2}R_3 - \left[-\frac{1}{2H} H_{ab} + \frac{1}{2H^2}H_aH_b \right]\big(\partial_\mu q^a\partial^\mu q^b - \epsilon \partial_\mu \hat{q}^a \partial^\mu \hat{q}^b \big) \\
 &- \frac{1}{H^2} \big(q^a\Omega_{ab}\partial_\mu q^b \big)^2 + \epsilon \frac{2}{H^2}\big(q^a\Omega_{ab}\partial_\mu \hat{q}^b \big)^2 \\
 &- \frac{1}{4H^2} \left(\partial_\mu \tilde{\phi} + 2\hat{q}^a \Omega_{ab} \partial_\mu \hat{q}^b \right)^2 .
\end{aligned}
\end{equation}
Lastly we simplify this by reintroducing the function $\tilde{H}$ and its corresponding metric $\tilde{H}_{ab}$ \cite{Mohaupt:2011aa}
\begin{equation}
\begin{aligned}
\label{eq:3dlag}
 \mathbf{e}_3^{-1} \La_3 &= - \frac{1}{2}R_3 - \tilde{H}_{ab}\big(\partial_\mu q^a\partial^\mu q^b - \epsilon \partial_\mu \hat{q}^a \partial^\mu \hat{q}^b \big) \\
 &- \frac{1}{H^2} \big(q^a\Omega_{ab}\partial_\mu q^b \big)^2 + \epsilon \frac{2}{H^2}\big(q^a\Omega_{ab}\partial_\mu \hat{q}^b \big)^2 \\
 &- \frac{1}{4H^2} \left(\partial_\mu \tilde{\phi} + 2\hat{q}^a \Omega_{ab} \partial_\mu \hat{q}^b \right)^2.
\end{aligned}
\end{equation}




\section{Supergravity in higher dimensions}
\label{sec:pbranes}

In this section, we give a limited introduction to supergravity in higher dimensions, effective to describe $p$-branes and their relationship to black hole solutions. This is a vast topic which deserves much more space than we give to it, and so we begin with a few references for a reader interested in the topic. A comprehensive overview is given in \cite{Youm:1997hw}, which covers everything and more of what we hope to describe. Marolf has a chapter in \cite{Horowitz:2012nnc} which is focused on how $p$-branes give rise to black hole solutions and is particularly readable. Duff has some TASI lecture notes on brane solutions from the perspective of AdS/CFT research \cite{Duff:1999rk}. Additional TASI lectures by Peet \cite{Peet:2000hn} and a review from ICTP by Stelle \cite{Stelle:1998xg} are based on BPS solutions and black holes from the perspective of string theory. Duff's article `Formally known as strings' offers a historical perspective on the history of higher-dimensional supergravity and its relationship to M-theory \cite{Duff:1996aw}.

This section first introduces supergravity in eleven dimensions and its relationship to string theory and M-theory. We then discuss the fundamental objects known as $p$-branes, which generalise the notion of the charged particle in four dimensions. We conclude the section by sketching how one can build black hole solutions which have ten and eleven dimension interpretations as intersecting brane configurations.

\subsection{Eleven-dimensional supergravity}

In Section \ref{sec:supersymmetry}, supergravity was introduced as the local description of supersymmetry, which itself was motivated as the extension of the Poincar\'e symmetry through the introduction of a \emph{fermionic} generator describing a symmetry transformation between fermions and bosons. In this context, it made sense to discuss supersymmetry in four dimensions, but we can consider the supersymmetry algebras in arbitrary dimension. Nahm proved in 1977 that if we wish to maintain that the highest spin states have spin-2, then the spacetime dimension is restricted to $D \leq 11$ \cite{Nahm:1977tg}. As a very rough sketch, we understand this from our discussions in four dimensions. In four dimensions, each spinor has four components and maintaining that the helicity $|\lambda| \leq 2$ forced at most 32 supercharges, setting $\N = 8$. In eleven dimensions, a single spinor has 32 components, and so maximal supersymmetry is given by $\N = 1$. If we were to try and raise the number of dimensions, we would find our spinors had at least 64 components (for $D = 12$), and so $D > 11$ is incompatible with our assumptions of highest spin states. 

More surprisingly, in 1978, Cremmer, Julia and Scherk proved that there is a unique eleven-dimensional supergravity theory described by the supergravity multiplet containing the graviton $G_{MN}$, the graviphoton $\psi_M$ and a three-form gauge potential $\mathcal{A}_{MNP}$, where the spacetime indices run from $M,N \in \{0,\ldots 10\}$. In this initial context, eleven-dimensional supergravity was thought of as a way to derive four-dimensional supergravity from dimensional reduction. It was this eleven-dimensional theory which bought new attention to the Kaluza-Klein reduction when the community started to try and find ways to wrap up the seven dimensions. In 1981, Witten produced a paper stating that the minimum number of dimensions to compactify over to get a standard-like model in four dimensions was seven \cite{Witten:1981me}. This seemed to only further motivate that eleven-dimensional supergravity was the `sweet-spot' for a unified theory.\footnote{This upper and lower bound for $D = 11$ from very different perspectives is described by Duff as what appears `to this day seems to be merely a gigantic coincidence' \cite{Duff:1999rk}.}

While supergravity was being worked on, Veneziano's dual model theory to describe the strong interaction \cite{Veneziano:1968yb} had wildly changed direction after being realised as a model of the relativistic string with the graviton appearing in the massless spectrum of the closed string. In 1984, the so-called `first superstring revolution' began when Green and Schwarz developed the type I superstring with the gauge group SO(32) \cite{Green:1984sg}; a theory free from gauge and gravitational anomalies. String theory became the new hope for a unification theory and for a few years, eleven-dimensional supergravity took a back seat. By 1985, there were five consistent string theories: the type I string already mentioned, two heterotic strings with gauge groups of either $E_8 \times E_8$ or SO(32) \cite{Gross:1985fr, Gross:1985rr} and two theories of closed strings called type IIA and type IIB \cite{Green:1981yb}. It was argued that one could break the maximal supersymmetry of the string in ten dimensions by reducing the heterotic string over a Calabi-Yau three fold, producing a theory with $\N = 1$ supersymmetry in four dimensions \cite{Candelas:1985en}. The new question was that if string theory was supposedly a `theory of everything', why were there five different theories. There was also the question of the non-perturbative realisation for the superstring, and other phenomenological questions such as how to make the choice of which Calabi-Yau to reduce over, and how can we understand the smallness of the cosmological constant \cite{Duff:1999rk}.

Eleven-dimensional supergravity made a surprising return to the limelight after Witten introduced a new, non-perturbative theory known as M-theory \cite{Witten:1995ex}. By appealing to a set of stringy dualities, it was put forward that one overarching theory in eleven dimensions could describe each of the five string theories as various points in a moduli space, with eleven-dimensional supergravity appearing as the low energy limit. 
\begin{equation*}
  \left.\begin{aligned}
  &E_8 \times E_8 \; \text{Heterotic string}\\
  &SO(32) \; \text{Heterotic string} \\
  &SO(32) \; \text{Type I string}\\
  &\text{Type IIA string}\\
  &\text{Type IIB string}\\
  &\text{Eleven-dimensional supergravity} \; \; \;\\
\end{aligned}\right\} \text{M-theory}
\end{equation*}
For the remainder of this section, we will focus on the low energy, bosonic field content of eleven-dimensional supergravity and its relationship to type IIA/IIB string theory, motivated that eleven-dimensional supergravity isn't only interesting in isolation but also through how M-theory makes contact with all consistent string theories. 

The low-energy action for the bosonic fields of eleven-dimensional supergravity is given by
\begin{equation}
\label{eq:11daction}
	S_{11D} = \frac{1}{2 \kappa_{11}^2} \int -R \star 1 - \half \star \mathcal{F} \wedge \mathcal{F} - \frac{1}{6} \mathcal{F} \wedge \mathcal{F} \wedge \mathcal{A}.
\end{equation}
Where $\mathcal{A}$ is the three-form gauge potential with the corresponding field strength $\mathcal{F} = d\mathcal{A}$ and $2\kappa^2_{11} = 16\pi G_{11}$ is the gravitational coupling. The fermionic completion of this action is given in \cite{Cremmer:1978km}. Although eleven dimensions is obviously something we're less used to, from the point of view of the action it's not that different to the Einstein-Maxwell action \eq{EMaction}, in which gravity is coupled to some two-form gauge field. This similarity will come up again when we consider $p$-branes shortly. Note that there is no scale dependent coupling, \ie there is no dilaton. 

Let us now consider the low energy, bosonic field content of the type II strings, which we will refer to as type IIA/IIB supergravity. The matter content of type IIA/IIB supergravity comes from two sectors known as the Neveu–Schwarz (NS-NS) sector and the Ramond (RR) sector, which get their names from the boundary conditions used from the string theory perspective. The NS-NS sectors of the type IIA and type IIB string theory are the same, whereas the RR sector is different for each. The bosonic content of the NS-NS sector is given by the graviton $g_{\mu \nu}$, a two-form potential $B_{\mu \nu}$, known as the Kalb-Ramond field, and the dilaton $\phi$. We can write the action as \cite{Witten:1995ex}
\begin{equation*}
	S_{\text{NS-NS}} = \frac{1}{2 \kappa_{10}^2} \int e^{-2\phi} \left( -R \star 1 + 4 d\phi \wedge \star d\phi - \half H \wedge \star H \right), \qquad H = dB.
\end{equation*}
where we note that this is written in the \emph{string-frame} where the Ricci scalar appears in the action coupled to the dilaton. This can be removed through a conformal transformation to obtain the action in the Einstein frame \cite{Polchinski:1998rr}
\begin{equation*}
	S_{\text{NS-NS}} = \frac{1}{2 \kappa_{10}^2} \int \left( -R \star 1 - \half d\phi \wedge \star d\phi - \half e^{-\phi} H \wedge \star H \right).
\end{equation*}
The RR sector is built from additional $p$-form field strengths. For the type IIA string, the gauge potentials are odd, with a one-form $A_1$ and three-form $A_3$. We can write down the low energy bosonic action for the RR sector \cite{Witten:1995ex}
\begin{equation*}
	S_{\text{IIA | RR}} = - \frac{1}{2 \kappa_{10}^2} \int F_2 \wedge \star F_2 +  \tilde{F}_4 \wedge \star \tilde{F}_4 - \frac{1}{4} F_4 \wedge F_4 \wedge B,
\end{equation*}
where the last term is a topological Chern-Simons term and the field strengths are given by
\begin{equation*}
	\begin{aligned}
		F_2 &= dA_1, \qquad \tilde{F}_4 = dA_3 + A_1 \wedge H + B \wedge F_4 \wedge F_4, \\
		F_4 &= dA_3.
	\end{aligned}
\end{equation*}

In contrast, the IIB string has a RR sector built from even-form gauge potentials: $A_0$ a scalar, $A_2$ a two-form gauge potential  and $A_4$ a four-form gauge potential, which has a self dual field strength $F_5$. The self-duality of $F_5$ prohibits a totally satisfactory Lagrangian, but if we are happy to impose this as an additional constraint, one can write \cite{DHoker:2002nbb}
\begin{equation*}
	S_{\text{IIB | RR}} = - \frac{1}{2 \kappa_{10}^2} \int F_1 \wedge \star F_1 +  \tilde{F}_3 \wedge \star \tilde{F}_3 + \half \tilde{F}_5 \wedge \star \tilde{F}_5 + A_4 \wedge H \wedge F_3,
\end{equation*}
where the various field strengths are given by
\begin{equation*}
	\begin{aligned}
		F_1 &= dA_0, \qquad  &&\tilde{F}_3 = F_3 - A_0 H, \\
		F_3 &= dA_2, \qquad &&\tilde{F}_5 = F_5 - \half A_2 \wedge H + \half B \wedge F_3, \\
		F_5 &= dA_4,  \qquad &&\tilde{F}_5 = \star	\tilde{F}_5, 
	\end{aligned}
\end{equation*}
where we note the last piece of the above is the self-duality constraint $\tilde{F}_5 = \star	\tilde{F}_5$, imposed by hand.

For the following work, we will not need the exact form of the actions, but we have collected them all together as many resources include only one or two. What we will be focused on is studying the fundamental objects of each theory by looking at how charged objects couple to the $p$-forms. 

As a closing remark, we comment on how these theories are related to one another. At the low energy level, we can build a duality map through dimensional compactification. Beginning with the eleven-dimensional theory, reducing \eq{11daction} over a compact, spacelike circle, one obtains the action for type IIA supergravity. It was first conjectured by Townsend in \cite{Townsend:1995kk} that the full type IIA string theory can be obtained by the dimensional compactification of the supermembrane in eleven dimensions. This is an example of the symmetry known as \emph{S-duality}, where the dilaton in the strong coupling limit behaves like an additional dimension \cite{Schwarz:1998tz}. Another S-duality appearing within string theory is the SL$(2,\mathbb{Z}$) invariance of type IIB string theory for theories related by the flipping of the string coupling $g_S$\cite{Sen:1994fa}.

The type IIA and type IIB supergravity theories can be shown to be dual through compactifying them both down to nine dimensions, where one can map the field content into each other. This duality is also seen at the level of string theory, which manifests as the stringy symmetry known as \emph{T-duality}, which is full equivalence at the level of the string partition function.  Roughly, T-duality is the statement that type IIA (IIB) compactified over a circle of radius $R$ is equivalent to type IIB (IIA) compactified over a circle of radius $R^{-1}$. In the process of this duality, the momentum and winding modes of the string are interchanged. 

\subsection{p-branes}

Given the actions of eleven-dimensional supergravity and the low energy descriptions of the type IIA/IIB strings, we can consider the fundamental objects of the various theories. The notion of a $p$-brane comes naturally as a generalisation of the charged particles of Maxwell theory, which is where we begin.

Maxwell theory contains a one-form potential $A$ which enters the action as a two-form field strength $F = dA$. Let us consider some charged particle that couples to the gauge potential in the following way
\begin{equation*}
	S_e = \mathcal{Q} \int A = \mathcal{Q} \int d\tau \; \left[ A_\mu  \diff{x^\mu}{\tau} \right].
\end{equation*}
The charge of the particle is found from the now familiar formula using Gauss' law:
\begin{equation*}
	\mathcal{Q} = \int_{S^2} \star F.
\end{equation*} 

As we saw in Section \ref{sec:electromagneticduality}, the electromagnetic duality allows one to define a dual gauge potential $\tilde{A}$ and we can consider the magnetic particle (magnetic monopole) that couples to $\tilde{A}$ analogously to the electric particle
\begin{equation*}
	S_m = \mathcal{P} \int \tilde{A} =\mathcal{P} \int d\tau \; \left[ \tilde{A}_\mu  \diff{x^\mu}{\tau} \right],
\end{equation*}
where the charge of the particle is found by integrating over the flux of the field
\begin{equation*}
	\mathcal{P} = \int_{S^2} F.
\end{equation*} 
We note that point-like magnetic charges have not been measured experimentally, but the symmetry we see here suggests that they exist, but with masses far beyond the reach of current experiments \cite{Becker:2007zj}. 

Let us now consider an identical system, but allow our Maxwell theory to be defined in $D$-dimensions. We have a one-form potential which will couple to a particle. However, to surround the particle in $D$ dimensions, we do not integrate over a $S^2$ but rather over $S^{D-2}$ such that the electric charge is found from
\begin{equation*}
	\mathcal{Q} = \int_{S^{D - 2}} \star F,
\end{equation*}
where we note that $\star F$ will be a $(D-2)$-form. Let us now consider the magnetic dual, which will have a charge equal to the integral over the flux of the field
\begin{equation*}
	\mathcal{P} = \int_{S^2} F.
\end{equation*}
We can no longer interpret this as a magnetic \emph{particle}, but rather as an object that extends into space. Working in $D$ dimensions, the sphere $S^2$ doesn't necessarily surround a point-like particle, but rather an extended object spanning $(D-4)$ spacelike dimensions. These extended objects are called $p$-branes. As some examples, we can think of a particle as a 0-brane and a string as a 1-brane. Taking the example of $D = 10$, we see that a particle that electrically couples to a one-form will have a magnetically dual 6-brane.

Let us now allow our discussion to be totally general, where our theory contains some gauge potential $A_{p+1}$ which is a $(p+1)$-form with a $F_{p+2} = dA_{p+1}$ field strength. By considering the charges of this gauge potential and its magnetic dual, we can understand the extended objects that couple to it. Our $(p+1)$-form gauge potential couples electrically to the world volume of a $p$-brane with $n = p+1$ spacetime dimensions.\footnote{This note is mainly for a student reading this getting used to the numbers. A particle has no extension in space, but moves through time. As such, we think of the particle as a 0-brane which has a worldvolume of one spacetime dimension. Generally we describe some theory with a gauge field which is an $(p+2)$-form, which has an $(p+1)$ gauge potential that couples to an extended object whose worldvolume has $(p+1)$ spacetime dimensions, which we call a $p$-brane. From this, the quick statement is given an $n-$form field strength (this is what appears in the action usually), one has a $(n-2)$-brane that couples electrically to its gauge potential.} We can write the action of the brane coupling to the $(p+1)$-form as
\begin{equation*}
	S = \mathcal{Q} \int A_{p+1},
\end{equation*}
where the electric charge of the $p$-brane is computed from
\begin{equation*}
	\mathcal{Q} = \int_{S^{D - p - 2}} \star F_{p+2}.
\end{equation*}
Similarly, the magnetic dual has a charge given by the integral over the flux
\begin{equation*}
	\mathcal{P} = \int_{S^{p+2}} F_{p+2}.
\end{equation*}
This charge can be understood as being sourced from a $(D-p-4)$-brane. Another way to see this is to consider the charge from the dual field with
\begin{equation*}
	\mathcal{\tilde{Q}} = \int_{S^{D - p - 2}} \star \tilde{F}_{D - p - 2}.
\end{equation*}
which is the electric charge of a $(D-p-2)$-form associated to a $(D-p-4)$-brane.

With this general rule, we can write down the extended objects of eleven-dimensional supergravity as well as type IIA and IIB supergravity. Note that these are not only objects in supergravity but also in M-theory and type IIA/IIB string theory \cite{Duff:1993ye}. 

For eleven-dimensional supergravity, we have only one gauge potential, which is the three-form $\mathcal{A}$, with the corresponding $2$-brane known as the M2 brane. This plays the role of the electron in four-dimensional Maxwell theory, coupling electrically to $\mathcal{A}$. The magnetic dual of the M2 brane is the $(11 - 2 - 4 = 5)$ 5-brane known as the M5 brane. We will discuss these in more detail in the next section, from the context of BPS solutions of supergravity.

In the type IIA/IIB supergravities, there are the NS-NS and RR sectors. In the NS-NS sector, we have the two-form $B$ which sources the electric 1-brane, known as the F1 string and its magnetic dual the NS5 brane. The F1 string is known as the \emph{fundamental string} and is the string of perturbative string theory. For type IIA, the RR gauge potentials are the one-form $A_1$ and the three-form $A_3$ which have the electric D0 particle and the D2 membrane coupling to them. Their magnetic duals are known as the D6 brane and D4 brane, respectively. Finally, the RR sector of the type IIB string has a zero-form $A_0$, two-form $A_2$ and four-form $A_4$. The two-form couples to the D1 string and its magnetic dual is the D5 brane. The electric source of the zero-form is sometimes called the D(-1) brane; when studied in Euclidean theories, we can consider this as an instanton solution in space and Euclidean time \cite{Becker:2007zj}. Its magnetic dual is the D7 brane. Finally the four-form couples both electrically and magnetically to the D3 brane as a result of the self-duality of the five-form $F_5 = dA_4 = \star F_5$. This is summarised in Table \ref{table:branes}. Here we have used words such as string and membrane for the lower dimensional branes to help with visualisation, but often all these extended objects are simply referred to as branes, e.g. the D1 brane or D0 brane.
 
\begin{table}[]
\centering
\def\arraystretch{1.3}
\begin{tabular}{|c|c|c|c|c|}
\hline
\multicolumn{1}{|c|}{Electric Brane} & \multicolumn{1}{c|}{D = 11} & \multicolumn{1}{c|}{Type IIA} & \multicolumn{1}{c|}{Type IIB} & \multicolumn{1}{c|}{Magnetic Dual} \\ \hline \hline
D(-1) Instanton                      & \textemdash & \textemdash   & $A_0$                         & D7 Brane                           \\ \hline
D0 Particle                          & \textemdash & $A_1$                         & \textemdash   & D6 Brane                           \\ \hline
F1 String                            & \textemdash & $B$                           & $B$                           & NS5 Brane                          \\
D1 String                            & \textemdash & \textemdash   & $A_2$                         & D5 Brane                           \\ \hline
M2 Membrane                          & $\mathcal{A}$               & \textemdash   & \textemdash   & M5 Brane                           \\
D2 Membrane                             & \textemdash & $A_3$                         & \textemdash   & D4 Brane                           \\ \hline
D3 Brane                             & \textemdash & \textemdash   & $A_4$                         & D3 Brane                           \\ \hline
\end{tabular}
\caption[Summary of the electric $p$-branes and their magnetic duals]{Summary of the electric $p$-branes and their magnetic duals \cite{DHoker:2002nbb}}
\label{table:branes}
\end{table}


\subsection{Extremal brane solutions}
\label{sec:extbranesol}

We now discuss $p$-branes from the perspective of solutions to the field equations. In particular, we are going to discuss the extremal solutions which preserve one half of the supersymmetry. As such, these extremal solutions are sometimes referred to as BPS solutions, or more accurately $\half-$BPS solutions. Solutions are found by making an ansatz for the $p$-brane with the symmetry of ISO$(1,p) \times \text{SO}(D-p-1)$, with the corresponding line element 
\begin{equation*}
	ds^2 = e^{2A(r)} dx^2_\parallel + e^{2B(r)} dx^2_\perp.
\end{equation*} 
We denote the coordinates parallel to the brane with $dx^2_\parallel = \eta_{\mu \nu} dy^\mu dy^\nu$, accounting for the symmetry group ISO$(1,p)$, and the directions transverse to the brane as $dx^2_{\perp} = \delta_{m n} dx^m dx^n$, accounting for the symmetries of the sphere. The variable $r$ is the isotropic coordinate corresponding to the transverse space: $r = \sqrt{x^m x_m}$. For the following discussion, we limit $p \leq 6$ to allow our solutions to be asymptotically flat. The inappropriate fall off of the so-called `large branes' is a common feature for brane solutions in arbitrary dimension where there are less than three transverse directions to the source, \eg black holes in three dimensions, or black strings in four dimensions \cite{Mohaupt:2000gc}.

We will not solve the field equations themselves, but rather write down the line elements with proper referencing so we can discuss their form. In  \cite{Stelle:1998xg, Duff:1999rk}, it is explained how to find brane solutions from the field equations in good detail. For our purposes, we wish to have these extremal $p$-brane solutions as building blocks for the next section, in which we look at the intersection of branes and their relationship to black hole solutions in lower dimensions. 

Let us begin by looking at the M2 and M5 brane solutions of eleven-dimensional supergravity. These are two of the four basic solutions of eleven-dimensional supergravity, with the other two being the PP-wave and the Kaluza-Klein monopole where the gauge field $\mathcal{A} = 0$. As these are also solutions in ten dimensions (where both the field strength and dilaton are assumed to be zero), we will cover them at the conclusion of this section.

Let us begin with the solution of the electric M2 brane  \cite{Duff:1990xz}, which is described by
\begin{equation*}
	ds^2 = H_2^{-\frac{2}{3}} dx^2_{\parallel} + H_2^{\frac{1}{3}} dx^2_\perp, \qquad \mathcal{A}_{\mu \nu \rho} = \es_{\mu \nu \rho} \; H_2^{-1}.
\end{equation*}
where $H_2$ is a harmonic function with respect to the transverse coordinates
\begin{equation*}
	\triangle^\perp H_2 = 0.
\end{equation*}
One particular choice we can make for the harmonic function corresponds to the single-centred solution
\begin{equation*}
	H_2(r) = 1 + \frac{Q_2}{r^6},
\end{equation*}
where $Q_2$ is related to the charge of the M2 brane. More generally, we write the harmonic function of a single centred solution with respect to the transverse coordinates of a $p$-brane solution as 
\begin{equation}
\label{eq:genericharmonic}
	H_p(r) = 1 + \frac{Q_p}{r^{D-p-3}}.
\end{equation}
Here $r$ is understood to be the isotropic coordinate as defined before, with the source located at $r \rightarrow 0$ and $Q_p$ is related to the charge of the brane. 

The global structure of the M2 brane can be understood in analogy to the extremal Reissner-Nordstr\"om solution of Einstein-Maxwell theory studied in Section \ref{sec:rnsol}. There is a Killing horizon located for $r \rightarrow 0$ and crossing the horizon, we find that the timelike Killing vector remains timelike and the singularity is therefore a timelike singularity. The conformal diagram is that of the extremal Reissner-Nordstr\"om (see Figure \ref{fig:PenroseERN}) but with each point on the diagram representing $\Real^7 \times S^2$ rather than $S^2$. We also see that just as the extremal Reissner-Nordstr\"om solution interpolated between $\M_4$ in the asymptotic limit to the Bertotti-Robinson solution $AdS_2 \times S^2$ near the horizon, the M2 brane interpolates between $\M_{11}$ in the asymptotic limit, to the near horizon geometry $AdS_4 \times S^7$. 

Let us now consider the M5 brane, which has a solution given by \cite{Gueven:1992hh}
\begin{equation*}
	ds^2 = H_5^{-\frac{1}{3}} dx^2_{\parallel} + H_5^{\frac{2}{3}} dx^2_\perp, \qquad \mathcal{F}_{\mu\nu\rho\sigma} = 3Q_5\es_{\mu\nu\rho\sigma}, \qquad H_5 = 1 + \frac{Q_5}{r^3}.
\end{equation*}
As with the M2 brane, we have a horizon located at $r \rightarrow 0$, and we find that the near horizon geometry is given by $AdS_7 \times S^4$. However, unlike the M2 brane, there is no spacetime singularity at all, and instead we find a copy of the spacetime on either side of the horizon \cite{Gibbons:1994vm}.

We might be worried that this solution is a counterexample for the Penrose singularity theorem, as we seem to have a trapping horizon bounding a region which contains no singularity. However, as the trapping horizon of the M5 brane solution is non-compact, a key assumption in Penrose's argument is broken, and so we should not expect the theorem to hold \cite{Horowitz:2012nnc}.

The D-brane solutions of type IIA and type IIB supergravity can be written down in a generic form, where specific solutions are obtained by setting the value of $p$ with the following string-frame line element \cite{Mohaupt:2000gc}
\begin{equation}
\label{eq:gendbrane}
\begin{aligned}
	ds^2 &= H_p^{-\frac{1}{2}} dx^2_{\parallel} + H_p^{\frac{1}{2}} dx^2_\perp, \qquad e^{\phi} =  H_p^{\frac{3-p}{4}}, \\
	A_p &= \begin{cases}
		 \frac{1}{H_p} dt \wedge dx^1 \wedge \ldots \wedge dx^p \qquad \; \; \; \quad p \leq 3, \\
		\star \left( \frac{1}{H_p} dt \wedge dx^1 \wedge \ldots \wedge  dx^p \right) \qquad p \geq 3,
	\end{cases}
\end{aligned}
\end{equation}
where $H_p$ is set by \eq{genericharmonic}. The presence of the dilaton can often lead to singular behaviour when dimensionally reducing the brane, and care must be taken to stabilise the scalar fields so they do not diverge on the horizon. We will postpone further discussion of the solutions until the next section, where we construct a four-dimensional black hole from an arrangement of intersecting branes from IIB supergravity.

The final two solutions are not brane solutions but are instead key parts of the construction of intersecting brane configurations. They come from solutions where the gauge field (and the dilaton for ten-dimensional theories) vanishes. As such, these can be considered as solutions of general relativity in higher dimensions. They are known as the PP-wave, or null particle solutions, and the Taub-NUT or Kaluza-Klein monopole solutions. 

The PP-wave, also sometimes called the Aichelburg-Sexl \cite{Aichelburg:1970dh} solution, can be found by infinitely boosting the \sch solution while scaling the mass $M$ such that the total energy $E$ is finite, and was first discussed by \cite{Brinkmann1925}. We can think of the PP-wave as being the solution describing the gravitational field of a null particle, and it is parameterised with a harmonic function $H_K$ with a line element given by
\begin{equation*}
ds^2 =	-dt^2 + dz^2 + (H_K - 1) (dt - dz)^2 + dx^2_{\perp},
\end{equation*}
where the perpendicular space has $(D-2)$ spacelike dimensions. The PP-wave solution can be superimposed along the direction of a brane such that we understand momentum modes moving in a direction along the brane. From the point of view of dimensional compactification, reducing over a direction with a PP-wave introduces an electric charge in the lower-dimensional theory, with the identification of the compact coordinate playing the role of charge quantisation.

The final solution we consider here is known as the Taub-NUT or Kaluza-Klein monopole solution, first considered as a solution in general relativity \cite{Taub:1951, osti:1963}, and then later interpreted in five dimensions, where it could be understood as a magnetic monopole solution after Kaluza-Klein compactification \cite{Sorkin:1987}, as well as an eleven-dimensional solution \cite{Han:1984ze}.

The line element of the Taub-NUT solution is given by
\begin{equation*}
\begin{aligned}
ds^2 &= -dt^2 + d\vec{y}^2 + H_{KK} dx^2  +H_{KK}^{-1} \left(d\theta + A_i dx^i \right)^2, \\
	H &= \phi = 0, \qquad \vec{\nabla} B = \vec{\nabla} \times \vec{A},
\end{aligned}	
\end{equation*}
and like the other solutions, is parameterised by a single harmonic function, which for this solution is a harmonic in the three-dimensional coordinates $x^i$. The spacetime coordinates: $-dt^2 + d\vec{y}^2$ cover $(D-4)$ dimensions and we can think of the spacetime having a decomposition as $\M^{D-4} \times K_4$ where $K_4$ is the Taub-NUT space with a line element
\begin{equation*}
	ds^2_{TN} = H_{KK}^{-1} \left(d\theta + A_i dx^i \right)^2, \qquad H_{KK} = 1 + \frac{K}{|r|}	.
\end{equation*}
We see that this naturally lends itself to compactification, with $\theta$ as the compactification dimension. It can be shown that like the BPS branes we discussed, the PP-wave and Taub-NUT solutions are also $\half$-BPS solution \cite{Stelle:1998xg}.

\subsection{Black holes and intersecting branes}
\label{sec:intersecting}

In this section, we outline the process of describing supersymmetric black hole solutions in lower dimensions as the dimensional compactification of the brane solutions given in the previous section. Given a $p$-brane solution in $D$ dimensions, we have two options for dimensional compactification. Reducing over a coordinate that runs parallel to the brane, we would obtain a $(p-1)$-brane in $(D-1)$ dimensions. This procedure is known as a double reduction or a wrapping. As the coordinates running parallel to the brane are isometries of the solution, this is always valid. The alternative reduction choice is to dimensionally reduce over a coordinate transverse to the brane. This produces a solution describing a $p$-brane in $(D-1)$ dimensions. However, generically the reduction direction will not be an isometry of the spacetime, as the harmonic functions that describe brane solutions are dependent on the isotropic coordinates built from the transverse directions. To perform the reduction, an isometry direction is established through smearing out the brane in a periodic array along the compactification direction. This generalises the harmonic function from a single source to that of a multi-centred solution. The branes can be placed arbitrarily within the array, with the gravitational and electromagnetic repulsions cancelling out due to the \emph{no-force} property of the BPS branes. To avoid going too far off-topic for the thesis, we do not detail this further but refer to the lecture notes \cite{Mohaupt:2000gc} which perform calculations and give examples on this topic. The upshot of this procedure is that the harmonic function then depends on all transverse directions except the reduction direction, which is then an isometry and a Kaluza-Klein reduction can be performed. These smeared brane configurations are often referred to as delocalised branes where the symmetry of the source is no longer spherical, but cylindrical. 

Constructing black holes from brane configurations is particularly interesting as from the higher-dimensional perspective, where we understand the microscopic origins of the M-branes and D-branes from M-/string theory. We can calculate the statistical entropy of the branes in higher dimensions and then compare it with the Bekenstein-Hawking area law of the black hole solutions in lower dimensions, opening a window into the quantum description of black hole thermodynamics \cite{Strominger:1996sh, Maldacena:1996ky,  Dabholkar:2004yr}. However, in this thesis we are interested in these black hole solutions from a reversed perspective, where we will show that the four-dimensional solutions we derive in Chapter \ref{ch:planarstu} can be described as higher-dimensional solutions through a process of dimensional lifting. In Chapter \ref{ch:brane}, we find the explicit solutions in ten and eleven dimensions which we can understand within the context of the brane solutions discussed in this section. As such, we include this section to give context for later discussions, and will not make further comments on entropy counting of brane configurations.

The main obstruction when finding black holes from the dimensional compactification of $p$-branes is their singular behaviour after reduction. For a generic $p$-brane solution, the result after compactification is degenerate, with a horizon of zero area leaving behind a null singularity in the lower-dimensional solution. This is a common issue for solutions with scalar fields which take singular values within the spacetime. When we solve our equations of motion in Chapter \ref{ch:planarstu}, we will have to carefully pick our integration constants such that our scalar fields do not diverge on the horizon. From the context of dimensional reduction, singular behaviour signifies that our solutions do not make sense from a lower-dimensional perspective.

In order to produce black hole solutions in lower dimensions, we then need a way to control the behaviour of the scalar fields. We can then understand the problem of finding regular solutions as the problem of `stabilising the moduli' \cite{Mohaupt:2000gc}. One way of doing this is to have the scalar functions given as ratios of harmonic functions, which can be achieved through introducing multiple $p$-branes into the spacetime which intersect along common directions. For the remainder of this section, we will discuss how to write down solutions of intersecting branes and give an example of the intersection of a D1 brane and D5 brane which produces a non-singular solution in four dimensions.

From the single brane solutions discussed in the previous section, it is possible to write down a set of generic rules to construct intersecting brane configurations that preserve some fraction of the supersymmetry. The steps we now offer are taken from an unpublished book by Klaus Behrndt \cite{Behrndt}, but the solutions we build will be given with published resources. The key part to understand is that the harmonic functions are solutions to Laplace's equation in the number of relative transverse directions to the whole configuration and so their fall off in terms of the isotropic coordinate depends on the brane system and not the constituent branes.

We will focus on solutions of intersecting branes that preserve some fraction of supersymmetry, which requires the number of relative transverse dimensions to be
\begin{equation*}
	n = 4 k, \qquad k \in \mathbb{Z},
\end{equation*}
which for $D \leq 11$ leaves us with $n = 0,4,8$. When $k = 1$, the fraction of supersymmetry preserved is one-quarter, which is what we will focus on in this discussion. When $n = 0$, we consider parallel branes, which include the stacked brane configurations which are commonly studied due to their relationship to the AdS/CFT correspondence \cite{Maldacena:1997re, Witten:1998qj, Maldacena:1996ky} which arguably has been the most celebrated and productive area of string theory and black hole physics for the past twenty-five years. For the case of $n = 8$, to obtain spherical black holes from dimensional reduction requires reducing over a non-flat space, and the resulting solutions are asymptotic to (anti)-de Sitter rather than Minkowski \cite{Behrndt}.

Let us now consider the rules for generating intersecting solutions. Considering the intersection of a $p$-brane and $q$-brane, the metric components multiply as
	\begin{equation*}
		g^{(p \times q)}_{\mu \nu} = g^{(p)}_{\mu \nu} g^{(q)}_{\mu \nu},
	\end{equation*}
	where we note that there's no summation over indices within this rule. The dilaton, if present, and gauge fields combine as
	\begin{equation*}
		\phi^{(p \times q)} = \phi^{(p)} + \phi^{(q)}, \qquad F^{(p \times q)} = F^{(p)} + F^{(q)},
	\end{equation*}
	where gauge fields will only combine when $p = q$, otherwise they are independent. For solutions with three or more intersecting branes, these rules described must be true for all pairs of branes. 
	
As a ten-dimensional example, lets construct the D1-D5 brane configuration \cite{Maldacena:1996ky} from the general solutions of D-branes in \eq{gendbrane} and the above rules
\begin{equation}
\begin{aligned}
\label{eq:d1d510d}
	ds^2_{1 \times 5} &= \frac{1}{\sqrt{H_1 H_5}} \left[-dt^2 + dz^2 \right] + \sqrt{\frac{H_1}{H_5}} \left[dy_1^2 + dy_2^2 + dy_3^2 + dy_4^2 \right]  + \sqrt{H_1 H_5} dx^2_\perp, \\
	e^{\phi} &= \sqrt{\frac{H_1}{H_5}}, \qquad H_i = 1 + \frac{Q_i}{r^2},
\end{aligned}
\end{equation}
where we notice that sending $Q_1 \rightarrow 0$ or $Q_5 \rightarrow 0$ will recover the solution for the D5 or D1 brane solution respectively, and we suppress the form of the gauge potentials which are independent from each other as they are different rank forms.

Another example we can study is the triple intersection of M5 branes \cite{Tseytlin:1996bh} which has a line element given by
\begin{equation}
\label{eq:triplem5}
\begin{aligned}
    ds_{5 \times 5 \times 5}^2 &= (H_1 H_2 H_3)^{-\frac{1}{3}} \big[ -dt^2 + dz^2 + H_1 H_2 H_3 d\vec{x}^{2} \\
    &+ H_1(dy_1^2 + dy_2^2) + H_2(dy_3^2 + dy_4^2) + H_3(dy_5^2 + dy_6^2) \big],
\end{aligned}
\end{equation}
where we now just use counting indices for the three M5 branes which have harmonics $H_i$.

We see that by intersecting the D1 and D5 brane, the coupling $e^{\phi}$ approaches a constant in the limit $r \rightarrow 0$. We can think of performing a dimensional reduction by wrapping over the coordinates $y_i$ in the solution \eq{d1d510d} to obtain the six-dimensional solution \cite{Behrndt}
\begin{equation*}
	ds^2_{6} = \frac{1}{\sqrt{H_1 H_5}} \left[-dt^2 + dz^2 \right] + \sqrt{H_1 H_5} \left(ds^2 + r^2 d\Omega^2_3 \right), \qquad 	e^{-2\phi} = 1, \qquad e^{2\sigma} = \sqrt{\frac{H_1}{H_5}}.
\end{equation*}
The ten-dimensional dilaton is balanced by the determinant of the internal space spanned by the $y_i$ coordinates and becomes a constant. The Kaluza-Klein scalar $\sigma$ parametrises the volume of the four torus. The wrapped D1-D5 solution is a dyonic string, with an electric charge sourced by the D1 brane, and magnetic charge from the D5 brane. To obtain a black hole solution (a 0-brane), we must reduce over the coordinate spanned by the string $z$. Performing this reduction, the metric written in the Einstein frame can be found to be \cite{Mohaupt:2000gc}
\begin{equation*}
	ds^2_5 = -\left(H_1 H_5\right)^{-\frac{2}{3}}  dt^2 + \left(H_1 H_5\right)^{\frac{1}{3}} \left(dr^2 + r^2 d\Omega_3^2 \right).
\end{equation*}
However, in the limit of $r \rightarrow 0$, the area of the event horizon tends towards zero. We can understand this singularity being sourced by $g_{zz} \rightarrow 0$ in the limit of $r \rightarrow 0$ from the six-dimensional perspective. We can interpret the vanishing of the area as the black string being in the ground state, and to stabilise the solution, we can include a PP-wave along the common intersection of the D1-D5 system, exciting the solution. Including a PP-wave along the string gives us a ten-dimensional solution \cite{Maldacena:1996ky}
\begin{equation}
\label{eq:triplem5wave}
\begin{aligned}
	ds^2_{1 \times 5 \times PP} &= \frac{1}{\sqrt{H_1 H_5}} \left[-dt^2 + dz^2 H_K(dz - dt)^2 \right] \\
	&+ \sqrt{\frac{H_1}{H_5}} \left[dy_1^2 + dy_2^2 + dy_3^2 + dy_4^2 \right] + \sqrt{H_1 H_5} dx^2_\perp.
\end{aligned}
\end{equation}
Reducing this over the five-torus we obtain a five-dimensional solution built from three charges \cite{Mohaupt:2000gc}
\begin{equation*}
	ds^2_5 = -\left(H_1 H_5 H_K\right)^{-\frac{2}{3}}  dt^2 + \left(H_1 H_5 H_K\right)^{\frac{1}{3}} \left(dr^2 + r^2 d\Omega_3^2 \right), \qquad H_i = 1 + \frac{Q_i}{r^2}.
\end{equation*}
This solution has a horizon of finite area, and the near horizon geometry is given by $AdS_2 \times S^3$ \cite{Behrndt}. In the special limit where all charges are equal, the solution simplifies to the Tangherlini solution \cite{Tangherlini:1963bw}
\begin{equation*}
	ds^2_5 = -H^{\; -2} dt^2 + H \left(dr^2 + r^2 d\Omega_3^2 \right), \qquad H = 1 + \frac{Q}{r^2},
\end{equation*}
which can be thought of as the five-dimensional equivalent to the four-dimensional extremal Reissner-Nordstr\"om solution. 

Finally, if we wish to obtain a four-dimensional black hole, we will have to perform one last reduction. However, the remaining spacelike coordinates are not generally isometries of the spacetime and so we will have to smear the solution before we can reduce it. The extended brane becomes a string, and our harmonic functions are restricted to depend only on three of the four coordinates in the relative transverse space $dx^2_\perp$. In the limit of $r \rightarrow 0$, we see that $\left(H_1 H_5 H_K\right)^{\frac{1}{3}}$ will diverge, and so following how we stabilised the previous reduction with the introduction of the PP-wave, we must add an additional charge before reducing to four dimensions. This can be achieved through assuming that the relative transverse directions cover the Taub-NUT space, rather than Minkowski. From a ten-dimensional perspective, we consider a D1-D5 brane intersection with a PP-wave and a four-dimensional space with a Kaluza-Klein monopole \cite{Behrndt}
\begin{equation}
\begin{aligned}
\label{eq:d1d5kkpp}
	ds^2_{1 \times 5 \times PP \times TN} &= \frac{1}{\sqrt{H_1 H_5}} \left[-dt^2 + dz^2 H_K(dz - dt)^2 \right] \\
	&+  \sqrt{\frac{H_1}{H_5}} \left[dy_1^2 + dy_2^2 + dy_3^2 + dy_4^2 \right] + \sqrt{H_1 H_5} \left[ \frac{1}{H_{KK}} (dx_4 + \vec{A} d\vec{x})^2 + H_{KK} d\vec{x}^2 \right].
\end{aligned}
\end{equation}
This can now be reduced over a six-torus to obtain a four-dimensional solution with four gauge fields \cite{Behrndt}
\begin{equation}
\label{eq:d1d5kkppred}
	\begin{aligned}
		ds^2 &= - \left(H_K H_1 H_{KK} H_5 \right)^{-\half} dt^2 + \left(H_K H_1 H_{KK} H_5 \right)^{\half} (dr^2 + r^2 d\Omega_2^2) ,\\
		F^1 &= d \left( H_1^{\; -1} \right) \wedge dt, \qquad F^3 = (\es_{\mu \nu \rho} \partial_\rho H_5) dx^\mu \wedge dx^\nu, \\
		F^2 &= d \left( H_K^{\; -1} \right) \wedge dt, \qquad F^4 = (\es_{\mu \nu \rho} \partial_\rho H_{KK}) dx^\mu \wedge dx^\nu \\
		e^{2\sigma_1} &= \frac{H_{K}}{H_1}, \qquad e^{2\sigma_2} =\frac{H_5}{H_{KK}}, \qquad e^{-2\phi} = \sqrt{\frac{H_{K} H_5}{H_1 H_{KK}}},
	\end{aligned}
\end{equation}
where we have two electric charges descending from the D1 brane and the PP wave, and two magnetic charges, descending from the D5 brane and the Kaluza-Klein monopole. The scalars $\sigma_i$ correspond to the radius of the internal torus from the reduction from six to four dimensions. The area of the horizon is finite, and the near horizon geometry is found to be $AdS_2 \times S^2$. This  solution can be thought of as a generalisation of the extremal Reissner-Nordstr\"om solution with four distinct gauge fields. 

We will not go through the details, but the same story can be played for the triple intersection of M5 branes \eq{triplem5}. By wrapping over the internal space, one obtains a five-dimensional solution with three magnetic charges corresponding to the three M5 branes. Obtaining a black hole in four dimensions by compactifying over the $z$ coordinate requires a stabilisation charge, which like the D1-D5 solution, is achieved through including PP-wave along common intersection \cite{Tseytlin:1996bh}
\begin{equation}
\begin{aligned}
    ds_{5 \times 5 \times 5 \times PP}^2 = &(H_1 H_2 H_3)^{-\frac{1}{3}} \big[ du dv + H_K du^2 + H_1 H_2 H_3 d\vec{x}^{2} \\
    &+ H_1(dy_1^2 + dy_2^2) + H_2(dy_3^2 + dy_4^2) + H_3(dy_5^2 + dy_6^2) \big].
\end{aligned}
\end{equation}
This can be reduced over a $T^7$ to obtain a line element in four dimensions \cite{Behrndt:1996jn}
\begin{equation}
\label{eq:triplem5red}
	ds^2_4 = -\left(H_K H_1 H_2 H_3 \right)^{-\frac{1}{2}} dt^2 + \left(H_K H_1 H_2 H_3 \right)^{\frac{1}{2}} (dr^2 + r^2 d\Omega^2_2),
\end{equation}
which again will have four distinct gauge fields and scalar fields corresponding to the volume of the internal torus produced though compactification. These four-dimensional, extremal solutions obtained from the compactification of intersecting BPS solutions in ten and eleven dimensions are related to extremal solutions of the STU model \cite{Behrndt:1996jn}. In Chapter \ref{ch:brane}, we will show that when taking the extremal limit of the higher-dimensional, non-extremal solutions we will encounter these intersections again.
