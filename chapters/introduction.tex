\chapter{Introduction}
\label{ch:introduction}
\pagenumbering{arabic}

Black holes are among the most wonderful and exotic objects to have come from theoretical physics. They appear naturally when we ask ourselves what happens when a body becomes so dense that the escape velocity is faster than the speed of light. In astrophysics, black holes appear as the final stage in the life cycle of certain stars, and in general relativity, they appear as some of the simplest solutions to Einstein's equations. In contemporary research, black holes have a central role in guiding us towards a consistent description of quantum gravity.

One of the most surprising mathematical results of general relativity is the appearance of curvature singularities from gravitational processes. In particular, Penrose's singularity theorem asserts that there is a curvature singularity within a trapping region of a spacetime \cite{Penrose:1964wq}; we can roughly understand this as the assertion that a singularity necessarily lives at the centre of a black hole. In 2020, Penrose received the Nobel prize in physics `for the discovery that black hole formation is a robust prediction of the general theory of relativity', a remarkable moment reflecting the growing acceptance from the wider academic community in the importance of mathematical research when understanding gravitation. 

Penrose's award is preceded by two experimental results which have bookended the work undertaken during our research. In 2016, LIGO and VIRGO announced their joint results measuring gravitational waves emitted from the merger of a binary black hole system \cite{Abbott:2016blz}. In 2019, the Event Horizon Telescope released a radio telescopic image of the shadow of a supermassive black hole \cite{Akiyama:2019cqa}. Extraordinarily, this publication not only spread through the scientific community, but the black hole's shadow was printed onto the front page of newspapers across the world. More than ever, black holes are being understood and accepted as physical objects outside of our imagination while still motivating some of our most abstract and theoretical work.

Classically, we understand a black hole as a region of spacetime from which nothing can escape. As such, the development of black hole thermodynamics is certainly one of the most surprising and interesting twists in the history of black hole physics. The story begins with Bekenstein's conjecture that a black hole has entropy proportional to its surface area \cite{Bekenstein:1973ur}, motivated by Hawking's area law that states that for any classical physical process, the area of a black hole weakly increases \cite{Hawking:1971vc}. Hawking's law sits within a group of four geometric laws known as the `laws of black hole mechanics' which are mathematically rigorous statements about black holes and Killing horizons.\footnote{A Killing horizon is a null hypersurface with a normal vector which is a solution to Killing's equation.} Hawking showed that for stationary black hole solutions, the event horizon of a black hole was a Killing horizon \cite{Hawking:1973uf}.

Not long after Bekenstein's conjecture, Hawking set the constant of proportionality by considering a quantum field theory in a curved spacetime background. Hawking was able to show that an observer external to the surface of a black hole would detect the emission of thermal radiation \cite{Hawking:1974sw} with a temperature\footnote{The Hawking temperature is proportional to the surface gravity $\kappa$ which for now can be understood as a constant associated to a Killing horizon.} and corresponding entropy given by 
 \begin{equation*}
 	T_H = \frac{\hbar \kappa }{2 \pi c k_B}, \qquad S_{BH} = \frac{k_B c^3}{\hbar G} \frac{A}{4}.
 \end{equation*}
 These relations are particularly beautiful in how they bring together the fundamental constants of special relativity: the speed of light ($c$), gravitation: Newton's constant ($G$), quantum mechanics: the reduced Planck's constant ($\hbar$) and statistical mechanics: Boltzmann's constant ($k_B$). Incredibly, by considering a black hole semi-classically, Hawking was able to show that the `region of no return' would emit energy and that eventually, the black hole would totally evaporate. The thermodynamic description of black holes led to the re-interpretation of the laws of black hole mechanics as the laws of thermodynamics. The parameters of a black hole solution, such as the mass, surface gravity and area now had thermodynamic interpretations as the internal energy, temperature and entropy respectively, and with this, a series of new thermodynamic and quantum mechanical questions could be asked of black holes.
 
In classical thermodynamic systems, when a system is considered macroscopically, the entropy measures the amount of energy in the system unable to do work. There is a second interpretation for thermodynamic systems from statistical mechanics, in which the entropy counts the number of microscopic configurations that would produce the same macroscopic system. The puzzle is then: if a black hole is a thermodynamic system, what are the microscopic degrees of freedom being counted by the Bekenstein-Hawking entropy? Another open problem is the \emph{information paradox} \cite{Hawking:1976ra}. The emission of Hawking radiation causes a black hole to lose energy and eventually totally evaporate. The particles emitted are purely thermal and so carry no information about the system. As such, we can imagine a pure state entering into a black hole region only to eventually be emitted, with all information about the original state having been lost. This loss of information violates unitarity and seems to be a fundamental incompatibility between black holes and quantum mechanics.
 
 The elephant in the room of contemporary theoretical physics is a consistent theory of quantum gravity. The beginning of the twentieth century brought about two incredibly fundamental theories: quantum mechanics and special relativity. The inconsistencies between Newtonian gravity and special relativity were immediately apparent, with the Newtonian potential predicting instantaneous changes. Led by the equivalence principle, Einstein put forward the theory of general relativity \cite{Einstein:1914bx, Einstein:1915ca}. Understanding gravitation as the curvature of a spacetime manifold gave new insight into experimental problems, such as providing the corrections to Mercury's orbit \cite{Einstein:1915bz}, but also predicted more exotic physics such as gravitational waves and black holes. However, despite the successes of general relativity, it is an effective field theory lacking the structure to explain gravitation in the smallest of scales.  
 
During the development of general relativity, there was a successful effort to incorporate special relativity into quantum mechanics, leading to quantum field theory. By the 1970s, the electromagnetic force, together with the strong and weak forces, had quantum field theory descriptions, and the unification of these theories produced the standard model of particle physics. The missing piece was gravity, which would not yield to a quantum field theory description due to unresolvable divergent behaviour when analysed perturbatively.

In our efforts to write down a theory for quantum gravity, we should expect black hole thermodynamics to play a central role in what questions we hope to be able to answer, and for the past fifty years, black holes have indeed been guiding our research. Most prominently, noticing that the entropy of a black hole was encoded by its area, rather than its volume, was the starting point for the holographic principle \cite{tHooft:1993dmi, Susskind:1994vu}, suggesting that a quantum description for gravity could be described as a boundary theory with one less dimension. This insight led to the development of the AdS/CFT conjecture \cite{Maldacena:1997re, Witten:1998qj} which has produced an incredible amount of work for the past twenty-five years, relating gravitational systems in $(D+1)$ dimensions, to quantum field theories in $D$ dimensions with a strong/weak coupling duality. 

Superstring theory \cite{Green:1987sp, Polchinski:1998rr} is the most prominent and hopeful candidate for a unified quantum theory. String theory describes particles as oscillation modes of a relativistic string and was initially put forward by Veneziano as a model for the scattering amplitudes of strongly interacting particles \cite{Veneziano:1968yb}. When a massless spin-two particle was found in the spectrum of the closed string, string theory began a new life as a possible theory of quantum gravity. In the early eighties, it was found that by including supersymmetry, the so-called superstring could be anomaly free \cite{Green:1981yb, Green:1984sg}, leading to the `first string revolution' and the realisation of a consistent, perturbative quantum theory including the graviton. However, there was still the problem of a non-perturbative description and black holes were out of string theory's scope. 

In the mid-nineties, the `second string revolution' began after the discovery of string dualities known as S- and T-duality, allowing us to understand the five once-distinct superstring theories as intimately related. It was shown that the strong coupling limit of one string theory was related to the weak coupling limit of another string theory, or eleven-dimensional supergravity, and with this insight came a way to study strongly coupled systems with a perturbative treatment of the dual theory. Furthermore, these dualities suggested that all superstring theories could be collected together into a single eleven-dimensional theory known as M-theory\footnote{What the `M' stands for in M-theory is a common footnote in texts on the subject, with suggestions such as matrix, magic, mystery, membrane and even mother suggested. Ultimately, no single answer is accepted (Witten offers three different answers up to the user's discretion). However, the silliest explanation I found is attributed to Sheldon Glashow, stating he wondered whether ```M' wasn't an upside-down `W' for Witten" \cite{holt_2006}.} with each theory appearing as an asymptotic expansion in different limits \cite{Witten:1995ex}.

Of particular interest when studying non-perturbative effects in string theory are a set of string soliton solutions known as $p$-branes. These $(p+1)$-dimensional objects can be understood as higher-dimensional analogues of charged particles and are fundamental objects of string theory. When a $p$-brane additionally preserves some fraction of supersymmetry, they are referred to as BPS solitons and are stable. When considered as solutions to the field equations, the BPS $p$-branes have a simple geometric description.  

In this thesis, we are particularly interested in charged black hole solutions of $\N = 2$ supergravity and their thermodynamics. Here, $\N$ counts the number of conserved supercurrents and in four dimensions, there are $4\N$ real supercharges. Although supergravity exists as the low-energy effective field theory of string theory, the study of supergravity predates string theory and instead was first studied by realising the super Poincar\'e symmetries of supersymmetry as local (or gauged) symmetries. From the point of view of studying black hole solutions and their thermodynamics, we can think of supergravity as the extension of general relativity with additional massless fields, the appearance of which allow for more complex systems of solutions and with that, the potential for more exotic black hole geometries. 

Since the bosonic sector of pure $\N=2$ supergravity is Einstein-Maxwell theory, the Reissner-Nordstr\"om solution of general relativity is additionally a solution to $\N = 2$ supergravity. The black hole is parameterised by its mass $M$ and charge $Q$, and provided that $M \geq Q$, the curvature singularity is hidden behind a Killing horizon. In the special case when $M = Q$, the Hawking temperature vanishes and the solution is said to be extremal. Viewed as a solution of $\N = 2$ supergravity, the extremal Reissner-Nordstr\"om solution is a BPS soliton. 

Charged black holes have a dual, perturbative description in terms of $p$-branes of type IIA and type IIB string theory \cite{Polchinski:1995mt}. The embedding of BPS black hole solutions into string theory allows us to return to the question of a microscopic understanding of a black hole's entropy. As the $p$-branes are supersymmetric configurations, it is possible to perturbatively compute various properties which hold for all values of the coupling. The entropy of the $p$-brane system can be computed in this way, and to leading order, it can be shown that this matches with the Bekenstein-Hawking area law \cite{Strominger:1996sh, Strominger:1996kf}.

The thermodynamic picture offered from string theory and its BPS configurations have the restriction that they necessarily describe black hole solutions with zero temperature.\footnote{Although near-extremal solutions can be studied with leading order approximations describing the greybody factors of Hawking radiation \cite{Maldacena:1996ix, Gubser:1998ex}.} Studying the full realisation of black hole thermodynamics requires non-zero temperature and so there are additional research questions from the supergravity perspective, searching for non-extremal solutions and comparing to known thermodynamic relationships. 

Of the four laws of black hole mechanics, the third law is the least understood from both the black hole and thermodynamic perspective. The law comes in two forms which are known as the strict and weak versions. In the strict version, it is said that the zero-temperature limit corresponds to a system with vanishing entropy, whereas the weak version only demands minimal entropy in the extremal limit. From a black hole perspective, the strict third law would impose that the black hole vanishes in the extremal limit, which generally is not the case; take for example the extremal charged black holes with dual string theory realisations. 

It is then an interesting question to search for black hole solutions in which the strict third law holds. Recently, a class of four-dimensional, non-extremal, planar symmetric black brane solutions of $\N = 2$ supergravity coupled to $n_V$ vector multiplets were presented \cite{Dempster:2015}. It was found in the extremal limit, the area density of the Killing horizon vanished, reproducing the so-called `Nernst branes' of \cite{Barisch:2011ui, Cardoso:2015wcf} and therefore were examples of a black hole solutions obeying the strict third law of thermodynamics.

Finding non-extremal solutions of supergravity theories is generally difficult. A common method to find black hole solutions in supergravity is by employing the Killing spinor equations; a set of first-order equations, generally easier to solve than Einstein's equations. The conditions placed by the Killing spinor equations then restrict the geometry of the solution which can lead towards the derivation of exact, analytic solutions. The price to pay is that the Killing spinor equations ensure the solutions are BPS states and so necessarily have zero Hawking temperature. 

The derivation of the non-extremal Nernst branes followed the work of \cite{Mohaupt:2011aa, Cortes:2015wca}, which developed the c-map in a new formulation using special real coordinates. The four-dimensional solutions are dimensionally reduced over a timelike circle, and in the real formulation of special geometry, the resulting equations of motion are symplectically covariant. By making non-trivial restrictions on the field configurations, it is possible to find exact Euclidean instanton solutions that can be lifted back into four dimensions and interpreted. Application of this procedure has led to a series of new non-extremal black hole solutions of $\N = 2$ supergravity \cite{Mohaupt:2011aa, Errington:2014bta, Dempster:2015, Dempster:2016}.

In this thesis, we use the real formulation of special K\"ahler geometry to derive new classes of solutions of $\N = 2$ supergravity, generalising the work on the Nernst branes by allowing there to be multiple charges. We find that these generalisations no longer obey the strict third law of thermodynamics, but instead yield a vastly different causal structure, containing external spacetime regions which are time-dependent. The Killing horizons of these solutions are understood to be cosmological horizons, and we are given the opportunity to study the laws of black hole mechanics in non-stationary spacetimes. By restricting our solutions, we recover solutions for Einstein-Maxwell theory, which can be understood as the Reissner-Nordstr\"om solution, but with planar rather than spherical symmetry. 

In the extremal limit, these cosmological solutions can be embedded into higher-dimensional supergravity, understood as $p$-brane configurations in ten or eleven dimensions. The cosmological solutions can be described thermodynamically, with an internal energy which is conserved and obeys the first law of thermodynamics. However, as the solution is non-stationary, the internal energy cannot be understood as a mass parameter as in conventional black hole thermodynamics. 

Considering $\N = 2$ supergravity theories related by flipping the sign of gauge coupling, planar symmetric solutions are derived. It is shown that these distinct solutions have the same partition functions, realising thermodynamically dual Killing horizons. Understanding that these solutions are derived from theories related by T-duality, we can look towards string theory for a deeper understanding of this relationship. Again, we find evidence of black hole thermodynamics highlighting relationships between distinct physical theories. The verification of the first law for our cosmological Killing horizons suggests a deeper and more fundamental thermodynamic interpretation for Killing horizons departing from the conventional solutions first considered fifty years ago. 

\section{Outline}

The main content of this thesis is structured into two parts. In Part \ref{part:background}, we concentrate on introducing the relevant background material needed to understand the research within this thesis. We split the background into three chapters. In Chapter \ref{ch:diff}, we introduce general relativity, assuming the reader is familiar with special relativity. The majority of the discussion is focused on the relevant differential geometry for the topic, and the conclusion of this chapter ties these areas together with the postulates of general relativity. We additionally include a discussion of the Lagrangian formulation of gravity. In Chapter \ref{ch:blackholes}, we discuss black holes in detail. We describe the geometry of Killing, trapping and event horizons. Using the \sch solution, we introduce the global structure of a black hole. We then use the Reissner-Nordstr\"om solution to give an in-depth discussion of deriving the black hole geometry from the field equations. We give a discussion of what it means to compute mass within general relativity and a general discussion of the laws of black hole mechanics and their relationship to thermodynamics. We conclude the chapter with the Euclidean action formalism, which is used to study the thermodynamic properties of the solutions within this thesis. In Chapter \ref{ch:supergravity}, we cover the necessary topics of supergravity. We begin by introducing supersymmetry through the extension of the Poincar\'e algebra. We build upon this, introducing the $\N = 2$ supergravity Lagrangians, the field content, and a few remarks on what we mean when we talk about supersymmetric black hole solutions. The electric-magnetic duality is discussed and its generalisation that appears for $\N = 2$ vector multiplet theories. Kaluza-Klein dimensional reduction is introduced and we use the double reduction of the STU model from six to four dimensions as an example. The c-map is then discussed, serving as a second example of dimensional reduction as well as introducing a key piece of the solution generating technique we use to find non-extremal solutions. Finally, we overview supergravity in higher dimensions, motivating it from the point of view of string/M-theory and discuss $p$-branes. We view these $p$-branes as solutions to the field equations and in particular we focus on the BPS solutions of $p$-branes and their intersections, relating this back to black hole solutions in lower dimensions.

In Part \ref{part:main}, we present the results of this thesis: planar symmetric solutions of the field equations for Einstein-Maxwell theory and the STU model of $\N = 2$ supergravity, and their corresponding thermodynamics. In Chapter \ref{ch:planarem}, we begin by making a planar symmetric, static ansatz for our geometry and find a solution of Einstein-Maxwell theory supported by an electric charge. Studying the static solution closely, we find that it contains a curvature singularity and a Killing horizon. We thus interpret the static patch as the \emph{interior} of our solution in analogy with the interior of the \sch solution, or alternatively, the region behind the Cauchy horizon of the Reissner-Nordstr\"om solution. Analytic continuation through the Killing horizon leads to a second region which we interpret as the \emph{exterior}. Here, the coordinates $\{t,r\}$ switch from timelike/spacelike to spacelike/timelike, and as such, the exterior region is dynamic (non-stationary). The explicit time-dependence of the exterior geometry leads to us naming the solution as a \emph{cosmological} solution. The remainder of the chapter is split between studying properties of the static region of the spacetime --- such as the motion of causal geodesics or the conserved charges of the solution --- and the global spacetime structure of a generalised class of cosmological solutions. Understanding the global structure leads to the classification of the horizons of the various solutions, which is vital for the discussion of the thermodynamics in later chapters. We conclude the chapter with a discussion of the extremal limit for the spacetime geometry. We notice that in the extremal limit, the location of the Killing horizon is `pushed off' to infinity; the static region becomes spacetime filling and the dynamic region is of zero size. We also find the area density of the solution diverges, indicating infinite entropy for the extremal solution.\footnote{This divergence occurs simultaneously with the horizon `vanishing' into the asymptotic distance, and so the physical interpretation of this divergence is not obvious.} The remaining spacetime contains a timelike singularity without a Killing horizon and so is understood as a naked singularity solution.

In Chapter \ref{ch:planarstu} we turn to study non-extremal, planar symmetric solutions of the STU model of $\N = 2$ supergravity. To solve the field equations, we begin with our four-dimensional theory and make an ansatz to impose staticity and planar symmetry. We then use the c-map, dimensionally reducing the solution over the timelike coordinate to obtain a three-dimensional Euclidean theory. Expressing this using the real formulation of special geometry, we write our field equations in a symplectically covariant manner. After making a restriction of the field content, we find an exact solution to the equations of motion and the Euclidean instanton solution is then uplifted back into four dimensions. Here, we impose regularity conditions on the solution to ensuring the existence of a Killing horizon with finite area density, and physical scalars with no divergent behaviour at the horizon. The resulting four-dimensional solutions are then studied through a series of coordinate changes and we find that qualitatively, the global structure of the solution is identical to that of the planar symmetric solutions of Einstein-Maxwell theory. We find that asymptotically the spacetime geometry is that of the Kasner type-D vacuum solution and that the extremal limit `undresses' the solution, removing the Killing horizon and leaving behind a static solution with a naked singularity. We conclude the chapter by showing that through making a specific choice in our integration constants, the physical scalars of the theory can be made constant, and the solution simplifies to reproduce the solution of the Einstein-Maxwell theory.

In Chapter \ref{ch:brane}, we study the extremal limit of these cosmological solutions from a different perspective. We begin by uplifting the four-dimensional, non-extremal solutions into higher dimensions, finding solutions to consistent truncations of five, six, ten and eleven-dimensional supergravity. From the perspective of ten and eleven dimensions, we find that after taking the four-dimensional extremal limit, we can describe the cosmological solutions as smeared brane configurations with only small departures from the canonical examples given in Chapter \ref{ch:supergravity}. In six dimensions, we find that by taking the extremal limit together with a charge balancing condition, we recover supersymmetric solutions despite having made no assumptions about supersymmetry while solving the equations of motion in the previous chapter.

In Chapter \ref{ch:triplewick}, we present our research on verifying the first law of thermodynamics for our planar symmetric, cosmological solutions. The first law is a differential relationship between the internal energy of the solution and the other thermodynamic quantities. The crux of our discussion is how to obtain a properly normalised mass-like parameter to vary, as the static region of the spacetime is finite, containing a singularity, and the external region of the spacetime is neither asymptotically flat nor stationary. We choose to employ the Euclidean action formalism, which is well suited to non-asymptotically flat solutions. The standard methodology of the Euclidean action formalism employs a Wick-rotation of the timelike coordinate, allowing us to study the Euclidean section of the geometry, and from a quantum-mechanical argument, we can derive a thermodynamic potential from the saddle-point approximation of the Euclidean action. From the thermodynamic potential, we can obtain an expression for the thermodynamic internal energy, which becomes the mass-like parameter we need. The issue we find for our classes of solutions is that a Wick-rotation within the static region does \emph{not} produce a smooth Euclidean geometry (due to the singularity) and a Wick-rotation of the timelike coordinate in the exterior region produces a complex line element. To work around this, we obtain a smooth, real Euclidean geometry from the exterior region of the solution through Wick-rotating all three spacelike coordinates. We refer to this technique as the \emph{triple Wick-rotation}. As a consistency check for this procedure, we use the de Sitter solution which contains a Killing horizon when written in static coordinates, but no singularity. We then use the Euclidean action formalism in both the static and dynamic regions of solution and verify the results are the same for both methods. We then turn to the planar symmetric solutions of this thesis to verify the first law of thermodynamics. 

One last complication we encounter is how to properly normalise the Euclidean action. When computing the Euclidean action, usually a term is included which corresponds to the subtraction of the background contribution. For solutions which are asymptotically flat, the solution can be considered as isolated and the background subtraction comes from a boundary term computed by embedding the solution into Minkowski space. More generally, there are commonly divergent contributions to the action when evaluating in the asymptotic limit. In these cases, the divergences can removed by including a counter term built from geometric data of the boundary manifold. In fact, for asymptotically flat solutions we find a divergence too and the Minkowski background is the appropriate counter term. Removing the divergences uniquely determines the background subtraction and hence the normalisation of the action. When evaluating the Euclidean action for our planar symmetric solutions, we find there is no natural background or divergent contribution, and hence no natural subtraction term. To remedy this, we introduce a  `boundary condition' which ensures that the electric charge computed from the thermodynamic partition function matches that of the conserved charge computed from Gauss' law. This condition sets an overall numerical normalisation for our partition function, and from this, we find a consistent formulation of the first law of thermodynamics and Smarr's law for both the solutions of Einstein-Maxwell theory and the STU model. At the end of the chapter, we then cover an alternative procedure known as the isolated horizon formalism. This method assumes the form of the first law and derives all thermodynamic quantities from horizon data. We find that this method is consistent with our work from the triple Wick-rotation.

Finally, in Chapter \ref{ch:conclusion}, we offer some thoughts on how this research can be continued, focusing on a particular project in which partial progress has been made. The thesis ends with a summary of what we have presented and a few closing thoughts. Part \ref{part:appendicies} is dedicated to the appendices of the thesis, containing our conventions and some additional discussions and calculational details relegated from the main text.