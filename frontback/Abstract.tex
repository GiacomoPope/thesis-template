%*******************************************************
% Abstract
%*******************************************************

\chapter*{Abstract}

We derive new classes of cosmological solutions of $\N = 2$ supergravity containing planar Killing horizons and develop a novel treatment of the Euclidean action formalism suitable for deriving a thermodynamic partition function for solutions with a time-dependent exterior region.

We consider solutions to Einstein-Maxwell theory and $\N = 2$ supergravity coupled to three vector multiplets, known as the \emph{STU model}. To obtain non-extremal solutions of the STU model, we solve the time-reduced field equations. Lifting back to four dimensions, the resulting static spacetime is incomplete, bounded by a curvature singularity on one side and a Killing horizon on the other. Analytic continuation reveals the existence of dynamic patches in the past and future, with the Kasner geometry recovered in the asymptotic limit. The global structure of the solutions to both Einstein-Maxwell theory and the STU model are shown to be the same. Restricting the integration constants in our solution to the STU model, we show that the scalar fields of the theory can be made constant, yielding the previously derived solutions of Einstein-Maxwell theory.

We find explicit lifts to five, six, ten and eleven dimensions which show that in the extremal limit, the underlying brane configuration is the same as for STU black holes. The extremal limit of the six-dimensional lift is shown to be BPS for special choices of the integration constants. We argue that there is a universal correspondence between spherically symmetric black hole solutions and planar cosmological solutions which can be illustrated using the Reissner-Nordstr\"om solution of Einstein-Maxwell theory.

We present a modified implementation of the Euclidean action formalism suitable for studying the thermodynamics of a class of cosmological solutions containing Killing horizons. To obtain a real metric of definite signature, we perform a \emph{triple Wick-rotation} by analytically continuing all spacelike directions. The resulting Euclidean geometry is used to calculate the Euclidean on-shell action which defines a thermodynamic potential. The thermodynamic potential obtained can be used to define an internal energy that obeys the first law of thermodynamics. Our approach is complementary to, but consistent with the isolated horizon formalism. 

We conclude with an outline of future work inspired by planar solutions in \emph{Einstein-anti-Maxwell} theory where the sign of the Maxwell coupling is flipped. These solutions are planar black holes rather than cosmological solutions. We show that upon a standard Wick-rotation, the black hole solutions give rise to the same Euclidean action and thermodynamic relations as the planar solutions of Einstein-Maxwell theory. We understand this as an indication of a thermodynamic duality between distinct theories. Considering  Einstein-anti-Maxwell theory as a consistent truncation of compactified type II$^*$ string theories, we propose that this duality can be generalised to the \emph{anti-STU model} where the sign of the gauge coupling matrix is flipped.		

\vfill